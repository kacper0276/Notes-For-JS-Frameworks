\documentclass[a4paper,12pt]{article}
\usepackage[utf8]{inputenc}
\usepackage{amsmath}
\usepackage{listings}
\usepackage{xcolor}
\usepackage{hyperref}
\usepackage{geometry}
\usepackage{pagecolor}

% Ustawienia ciemnego motywu
\definecolor{darkbackground}{HTML}{2E3440}
\definecolor{lighttext}{HTML}{D8DEE9}
\definecolor{keywordcolor}{HTML}{81A1C1}
\definecolor{stringcolor}{HTML}{A3BE8C}
\definecolor{commentcolor}{HTML}{616E88}
\definecolor{numbercolor}{HTML}{B48EAD}
\definecolor{functioncolor}{HTML}{88C0D0}

\pagecolor{darkbackground}
\color{lighttext}
\geometry{margin=1in}

% Ustawienia kolorowania składni kodu
\lstset{
  backgroundcolor=\color{darkbackground},
  basicstyle=\ttfamily\color{lighttext},
  keywordstyle=\color{keywordcolor},
  stringstyle=\color{stringcolor},
  commentstyle=\color{commentcolor}\textit,
  numberstyle=\tiny\color{numbercolor},
  identifierstyle=\color{functioncolor},
  breaklines=true,
  frame=single,
  rulecolor=\color{lighttext},
  tabsize=2,
  showstringspaces=false,
  captionpos=b,
}

\title{Zaawansowany Przewodnik po Node.js}
\author{Kacper Renkel}
\date{\today}

\begin{document}

\maketitle

\tableofcontents
\newpage

\section{Node.js}
Node.js to środowisko uruchomieniowe JavaScript, które umożliwia uruchamianie kodu JavaScript na serwerze. Oferuje wiele funkcji i bibliotek do budowy aplikacji serwerowych.

\section{Kontrolery}
W Node.js, kontrolery to zazwyczaj funkcje lub klasy, które obsługują różne trasy w aplikacji. W Express.js, można je zdefiniować jako middleware.

\subsection{Przykład Kontrolera w Express.js}
\begin{lstlisting}[language=JavaScript, caption=Przykład kontrolera w Express.js]
const express = require('express');
const router = express.Router();

router.get('/users/:id', (req, res) => {
  // Kod do pobrania użytkownika
  res.send(`User with ID: ${req.params.id}`);
});

router.post('/users', (req, res) => {
  // Kod do tworzenia użytkownika
  res.send('User created');
});

module.exports = router;
\end{lstlisting}

\section{Serwisy}
Serwisy w Node.js są klasami lub funkcjami odpowiedzialnymi za logikę biznesową aplikacji. Zwykle są używane do komunikacji z bazą danych lub innymi zewnętrznymi źródłami danych.

\subsection{Przykład Serwisu}
\begin{lstlisting}[language=JavaScript, caption=Przykład serwisu w Node.js]
class UserService {
  constructor(userModel) {
    this.userModel = userModel;
  }

  async getUser(id) {
    return this.userModel.findById(id);
  }

  async createUser(userData) {
    return this.userModel.create(userData);
  }
}

module.exports = UserService;
\end{lstlisting}

\section{Repozytoria}
Repozytoria w Node.js służą do zarządzania interakcją z bazą danych i często są tworzone przy użyciu ORM, takich jak Sequelize lub Mongoose.

\subsection{Przykład Repozytorium z Sequelize}
\begin{lstlisting}[language=JavaScript, caption=Przykład repozytorium z Sequelize]
const { Model, DataTypes } = require('sequelize');
const sequelize = require('./database'); // Konfiguracja bazy danych

class User extends Model {}

User.init({
  username: DataTypes.STRING,
  password: DataTypes.STRING,
}, {
  sequelize,
  modelName: 'User',
});

module.exports = User;
\end{lstlisting}

\section{JWT (JSON Web Token)}
JWT jest używany w Node.js do autoryzacji i uwierzytelniania. Używa się bibliotek takich jak `jsonwebtoken` do generowania i weryfikowania tokenów.

\subsection{Przykład Generowania i Weryfikowania Tokenu}
\begin{lstlisting}[language=JavaScript, caption=Przykład użycia JWT w Node.js]
const jwt = require('jsonwebtoken');

function generateToken(payload) {
  return jwt.sign(payload, 'your-secret-key', { expiresIn: '1h' });
}

function verifyToken(token) {
  return jwt.verify(token, 'your-secret-key');
}

module.exports = { generateToken, verifyToken };
\end{lstlisting}

\section{Hashowanie Haseł}
Do hashowania haseł w Node.js można użyć biblioteki `bcrypt`.

\subsection{Przykład Hashowania Haseł}
\begin{lstlisting}[language=JavaScript, caption=Przykład hashowania haseł przy użyciu bcrypt]
const bcrypt = require('bcrypt');

async function hashPassword(password) {
  const salt = await bcrypt.genSalt();
  return bcrypt.hash(password, salt);
}

async function comparePassword(password, hash) {
  return bcrypt.compare(password, hash);
}

module.exports = { hashPassword, comparePassword };
\end{lstlisting}

\section{Middleware}
Middleware w Node.js to funkcje, które mogą modyfikować żądania i odpowiedzi lub przeprowadzać dodatkowe operacje przed dotarciem do końcowego kontrolera.

\subsection{Przykład Middleware}
\begin{lstlisting}[language=JavaScript, caption=Przykład middleware w Node.js]
function loggerMiddleware(req, res, next) {
  console.log(`Request made to: ${req.url}`);
  next();
}

module.exports = loggerMiddleware;
\end{lstlisting}

\section{Guardy}
W Node.js, implementacja zabezpieczeń jest często realizowana za pomocą middleware. Middleware działa podobnie do guardów w innych frameworkach.

\subsection{Przykład Guardu (Middleware) w Node.js}
\begin{lstlisting}[language=JavaScript, caption=Przykład guardu w Node.js]
function authGuard(req, res, next) {
  const token = req.headers['authorization'];
  if (token === 'valid-token') {
    next();
  } else {
    res.status(403).send('Forbidden');
  }
}

module.exports = authGuard;
\end{lstlisting}

\section{Interceptory}
W Node.js, podobne funkcje można zrealizować poprzez middleware, które modyfikuje żądania lub odpowiedzi.

\subsection{Przykład Interceptora (Middleware) w Node.js}
\begin{lstlisting}[language=JavaScript, caption=Przykład interceptora w Node.js]
function responseInterceptor(req, res, next) {
  console.log('Before handling request');
  res.on('finish', () => {
    console.log('After handling request');
  });
  next();
}

module.exports = responseInterceptor;
\end{lstlisting}

\section{Protected Routes}
Protected routes w Node.js są zabezpieczane za pomocą middleware, które sprawdza autoryzację przed dostępem do chronionych zasobów.

\subsection{Przykład Chronionej Trasy}
\begin{lstlisting}[language=JavaScript, caption=Przykład chronionej trasy w Node.js]
const express = require('express');
const app = express();
const authGuard = require('./authGuard');

app.use('/protected', authGuard, (req, res) => {
  res.send('This is a protected route');
});

app.listen(3000, () => {
  console.log('Server is running on port 3000');
});
\end{lstlisting}

\section{Łączenie się z Bazą Danych}
Node.js obsługuje różne biblioteki ORM/ODM do komunikacji z bazą danych. Poniżej przedstawiam konfigurację dla dwóch popularnych bibliotek: Sequelize (dla SQL) oraz Mongoose (dla MongoDB).

\subsection{Sequelize}
Sequelize to popularny ORM dla Node.js, który obsługuje różne bazy danych SQL, takie jak MySQL, PostgreSQL, SQLite.

\subsubsection{Przykład Konfiguracji Sequelize}
\begin{lstlisting}[language=JavaScript, caption=Konfiguracja Sequelize w Node.js]
const { Sequelize } = require('sequelize');

const sequelize = new Sequelize('database', 'username', 'password', {
  host: 'localhost',
  dialect: 'mysql', // lub 'postgres', 'sqlite', 'mssql'
});

module.exports = sequelize;
\end{lstlisting}

\subsection{Mongoose}
Mongoose to popularne ODM dla MongoDB. Pozwala na łatwe połączenie z bazą danych MongoDB i zarządzanie schematami danych.

\subsubsection{Przykład Konfiguracji Mongoose}
\begin{lstlisting}[language=JavaScript, caption=Konfiguracja Mongoose w Node.js]
const mongoose = require('mongoose');

mongoose.connect('mongodb://localhost/nest', { useNewUrlParser: true, useUnifiedTopology: true })
  .then(() => console.log('Connected to MongoDB'))
  .catch(err => console.error('Failed to connect to MongoDB', err));

module.exports = mongoose;
\end{lstlisting}

\end{document}
