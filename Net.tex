\documentclass[a4paper,12pt]{article}
\usepackage[utf8]{inputenc}
\usepackage{amsmath}
\usepackage{listings}
\usepackage{xcolor}
\usepackage{hyperref}
\usepackage{geometry}
\usepackage{pagecolor} % Pakiet do zmiany koloru tła

% Ustawienia ciemnego motywu
\definecolor{darkbackground}{HTML}{2E3440}
\definecolor{lighttext}{HTML}{D8DEE9}
\definecolor{keywordcolor}{HTML}{81A1C1}
\definecolor{stringcolor}{HTML}{A3BE8C}
\definecolor{commentcolor}{HTML}{616E88}
\definecolor{numbercolor}{HTML}{B48EAD}
\definecolor{functioncolor}{HTML}{88C0D0}

\pagecolor{darkbackground}
\color{lighttext}
\geometry{margin=1in}

% Ustawienia kolorowania składni kodu
\lstset{
  backgroundcolor=\color{darkbackground},
  basicstyle=\ttfamily\color{lighttext},
  keywordstyle=\color{keywordcolor},
  stringstyle=\color{stringcolor},
  commentstyle=\color{commentcolor}\textit,
  numberstyle=\tiny\color{numbercolor},
  identifierstyle=\color{functioncolor},
  breaklines=true,
  frame=single,
  rulecolor=\color{lighttext},
  tabsize=2,
  showstringspaces=false,
  captionpos=b,
}

\title{Przewodnik po .NET}
\author{Kacper Renkel}
\date{\today}

\begin{document}

\maketitle

\tableofcontents
\newpage

\section{Wprowadzenie}
.NET to platforma programistyczna stworzona przez Microsoft, która pozwala na tworzenie aplikacji na różne platformy. W tym dokumencie omówimy najważniejsze komponenty w architekturze aplikacji .NET, takie jak kontrolery, serwisy, repozytoria, middleware oraz bezpieczeństwo za pomocą JWT.

\section{Kontrolery}
Kontrolery w .NET odpowiadają za obsługę żądań HTTP i zwracanie odpowiednich odpowiedzi. Zazwyczaj są to klasy, które dziedziczą po \texttt{ControllerBase} lub \texttt{Controller}.

\begin{lstlisting}[language=C#, caption=Przykład kontrolera w .NET]
using Microsoft.AspNetCore.Mvc;

[ApiController]
[Route("api/[controller]")]
public class UsersController : ControllerBase
{
    private readonly IUserService _userService;

    public UsersController(IUserService userService)
    {
        _userService = userService;
    }

    [HttpGet("{id}")]
    public IActionResult GetUser(int id)
    {
        var user = _userService.GetUserById(id);
        if (user == null)
        {
            return NotFound();
        }
        return Ok(user);
    }
}
\end{lstlisting}

\section{Serwisy}
Serwisy w .NET służą do enkapsulacji logiki biznesowej. Są one rejestrowane w kontenerze Dependency Injection (DI) i wstrzykiwane do kontrolerów.

\begin{lstlisting}[language=C#, caption=Przykład serwisu w .NET]
public interface IUserService
{
    User GetUserById(int id);
}

public class UserService : IUserService
{
    private readonly IUserRepository _userRepository;

    public UserService(IUserRepository userRepository)
    {
        _userRepository = userRepository;
    }

    public User GetUserById(int id)
    {
        return _userRepository.FindById(id);
    }
}
\end{lstlisting}

\section{Repozytoria}
Repozytoria w .NET są odpowiedzialne za komunikację z bazą danych. Służą do abstrakcji operacji na danych, co ułatwia testowanie oraz utrzymanie aplikacji.

\begin{lstlisting}[language=C#, caption=Przykład repozytorium w .NET]
public interface IUserRepository
{
    User FindById(int id);
}

public class UserRepository : IUserRepository
{
    private readonly ApplicationDbContext _context;

    public UserRepository(ApplicationDbContext context)
    {
        _context = context;
    }

    public User FindById(int id)
    {
        return _context.Users.Find(id);
    }
}
\end{lstlisting}

\section{Łączenie z bazą danych}
W .NET najczęściej używa się Entity Framework (EF) do komunikacji z bazą danych. Konfiguracja połączenia z bazą danych odbywa się w \texttt{Startup.cs} lub \texttt{Program.cs}.

\begin{lstlisting}[language=C#, caption=Konfiguracja Entity Framework]
public class Startup
{
    public void ConfigureServices(IServiceCollection services)
    {
        services.AddDbContext<ApplicationDbContext>(options =>
            options.UseSqlServer(Configuration.GetConnectionString("DefaultConnection")));
    }
}
\end{lstlisting}

\section{JWT (JSON Web Token)}
JWT jest standardem bezpiecznego przesyłania informacji pomiędzy stronami jako obiekt JSON. W .NET JWT są często używane do autoryzacji użytkowników.

\subsection{Konfiguracja JWT}
Aby skonfigurować JWT w .NET, należy dodać odpowiednie usługi w \texttt{Startup.cs} lub \texttt{Program.cs}.

\begin{lstlisting}[language=C#, caption=Konfiguracja JWT]
public void ConfigureServices(IServiceCollection services)
{
    services.AddAuthentication(options =>
    {
        options.DefaultAuthenticateScheme = JwtBearerDefaults.AuthenticationScheme;
        options.DefaultChallengeScheme = JwtBearerDefaults.AuthenticationScheme;
    })
    .AddJwtBearer(options =>
    {
        options.TokenValidationParameters = new TokenValidationParameters
        {
            ValidateIssuer = true,
            ValidateAudience = true,
            ValidateLifetime = true,
            ValidateIssuerSigningKey = true,
            ValidIssuer = Configuration["Jwt:Issuer"],
            ValidAudience = Configuration["Jwt:Audience"],
            IssuerSigningKey = new SymmetricSecurityKey(Encoding.UTF8.GetBytes(Configuration["Jwt:Key"]))
        };
    });

    services.AddControllers();
}
\end{lstlisting}

\subsection{Generowanie JWT}
Aby wygenerować JWT, należy stworzyć metodę, która stworzy token na podstawie danych użytkownika.

\begin{lstlisting}[language=C#, caption=Przykład generowania JWT]
public string GenerateJwtToken(User user)
{
    var securityKey = new SymmetricSecurityKey(Encoding.UTF8.GetBytes(_configuration["Jwt:Key"]));
    var credentials = new SigningCredentials(securityKey, SecurityAlgorithms.HmacSha256);

    var claims = new[]
    {
        new Claim(JwtRegisteredClaimNames.Sub, user.Username),
        new Claim(JwtRegisteredClaimNames.Email, user.Email),
        new Claim(JwtRegisteredClaimNames.Jti, Guid.NewGuid().ToString())
    };

    var token = new JwtSecurityToken(
        issuer: _configuration["Jwt:Issuer"],
        audience: _configuration["Jwt:Audience"],
        claims: claims,
        expires: DateTime.Now.AddMinutes(30),
        signingCredentials: credentials);

    return new JwtSecurityTokenHandler().WriteToken(token);
}
\end{lstlisting}

\section{Hashowanie haseł}
W .NET do hashowania haseł można użyć wbudowanego narzędzia \texttt{PasswordHasher}, które jest częścią biblioteki \texttt{Microsoft.AspNetCore.Identity}.

\begin{lstlisting}[language=C#, caption=Przykład hashowania hasła]
public class UserService : IUserService
{
    private readonly IUserRepository _userRepository;
    private readonly IPasswordHasher<User> _passwordHasher;

    public UserService(IUserRepository userRepository, IPasswordHasher<User> passwordHasher)
    {
        _userRepository = userRepository;
        _passwordHasher = passwordHasher;
    }

    public void RegisterUser(User user, string password)
    {
        user.PasswordHash = _passwordHasher.HashPassword(user, password);
        _userRepository.Add(user);
    }
}
\end{lstlisting}

\section{Middlewary}
Middlewary w .NET to komponenty, które są używane do przetwarzania żądań HTTP na różnych etapach ich cyklu życia. Można je wykorzystać np. do logowania, autoryzacji lub obsługi błędów.

\begin{lstlisting}[language=C#, caption=Przykład tworzenia middleware]
public class LoggingMiddleware
{
    private readonly RequestDelegate _next;

    public LoggingMiddleware(RequestDelegate next)
    {
        _next = next;
    }

    public async Task InvokeAsync(HttpContext context)
    {
        Console.WriteLine($"Request for {context.Request.Path} received at {DateTime.Now}");
        await _next(context);
        Console.WriteLine($"Response for {context.Request.Path} sent at {DateTime.Now}");
    }
}

// W Startup.cs
public void Configure(IApplicationBuilder app, IHostingEnvironment env)
{
    app.UseMiddleware<LoggingMiddleware>();
    app.UseRouting();
    app.UseEndpoints(endpoints =>
    {
        endpoints.MapControllers();
    });
}
\end{lstlisting}

\section{Guardy}
Guardy to mechanizmy w .NET, które można używać do ochrony określonych zasobów przed nieautoryzowanym dostępem. Można je implementować za pomocą polityk autoryzacyjnych.

\begin{lstlisting}[language=C#, caption=Przykład implementacji guardu]
public class AdminGuard : IAuthorizationRequirement
{
}

public class AdminGuardHandler : AuthorizationHandler<AdminGuard>
{
    protected override Task HandleRequirementAsync(AuthorizationHandlerContext context, AdminGuard requirement)
    {
        if (context.User.IsInRole("Admin"))
        {
            context.Succeed(requirement);
        }
        return Task.CompletedTask;
    }
}

// W Startup.cs
public void ConfigureServices(IServiceCollection services)
{
    services.AddAuthorization(options =>
    {
        options.AddPolicy("AdminOnly", policy => policy.Requirements.Add(new AdminGuard()));
    });
    services.AddSingleton<IAuthorizationHandler, AdminGuardHandler>();
}
\end{lstlisting}

\section{Inne ważne aspekty}

\subsection{Konfiguracja Dependency Injection (DI)}
Dependency Injection (DI) to mechanizm pozwalający na wstrzykiwanie zależności do komponentów takich jak serwisy, kontrolery czy repozytoria.

\begin{lstlisting}[language=C#, caption=Przykład konfiguracji DI w Startup.cs]
public void ConfigureServices(IServiceCollection services)
{
    services.AddTransient<IUserService, UserService>();
    services.AddScoped<IUserRepository, UserRepository>();
    services.AddSingleton<ILogger, Logger>();
}
\end{lstlisting}

\subsection{Obsługa wyjątków globalnych}
W .NET można skonfigurować globalną obsługę wyjątków za pomocą middleware.

\begin{lstlisting}[language=C#, caption=Przykład globalnej obsługi wyjątków]
public class ExceptionMiddleware
{
    private readonly RequestDelegate _next;

    public ExceptionMiddleware(RequestDelegate next)
    {
        _next = next;
    }

    public async Task InvokeAsync(HttpContext context)
    {
        try
        {
            await _next(context);
        }
        catch (Exception ex)
        {
            await HandleExceptionAsync(context, ex);
        }
    }

    private Task HandleExceptionAsync(HttpContext context, Exception exception)
    {
        context.Response.ContentType = "application/json";
        context.Response.StatusCode = (int)HttpStatusCode.InternalServerError;

        var result = JsonConvert.SerializeObject(new { error = exception.Message });
        return context.Response.WriteAsync(result);
    }
}

// W Startup.cs
public void Configure(IApplicationBuilder app)
{
    app.UseMiddleware<ExceptionMiddleware>();
    app.UseRouting();
    app.UseEndpoints(endpoints =>
    {
        endpoints.MapControllers();
    });
}
\end{lstlisting}

\subsection{Swagger - dokumentacja API}
Swagger to narzędzie do automatycznego generowania dokumentacji API. W .NET można je łatwo skonfigurować.

\begin{lstlisting}[language=C#, caption=Konfiguracja Swagger w Startup.cs]
public void ConfigureServices(IServiceCollection services)
{
    services.AddSwaggerGen(c =>
    {
        c.SwaggerDoc("v1", new OpenApiInfo { Title = "My API", Version = "v1" });
    });
}

public void Configure(IApplicationBuilder app)
{
    app.UseSwagger();
    app.UseSwaggerUI(c =>
    {
        c.SwaggerEndpoint("/swagger/v1/swagger.json", "My API V1");
    });
}
\end{lstlisting}

\section{Podsumowanie}
.NET jest wszechstronną platformą, która pozwala na tworzenie rozbudowanych aplikacji webowych z użyciem nowoczesnych wzorców projektowych. Dobre zrozumienie takich komponentów jak serwisy, repozytoria, middleware czy mechanizmy autoryzacji i autentykacji pozwala na budowanie skalowalnych i bezpiecznych aplikacji.

\end{document}
