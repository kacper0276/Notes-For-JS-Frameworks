\documentclass[a4paper,12pt]{article}
\usepackage[utf8]{inputenc}
\usepackage{amsmath}
\usepackage{listings}
\usepackage{xcolor}
\usepackage{hyperref}
\usepackage{geometry}
\usepackage{pagecolor}

% Ustawienia ciemnego motywu
\definecolor{darkbackground}{HTML}{2E3440}
\definecolor{lighttext}{HTML}{D8DEE9}
\definecolor{keywordcolor}{HTML}{81A1C1}
\definecolor{stringcolor}{HTML}{A3BE8C}
\definecolor{commentcolor}{HTML}{616E88}
\definecolor{numbercolor}{HTML}{B48EAD}
\definecolor{functioncolor}{HTML}{88C0D0}

\pagecolor{darkbackground}
\color{lighttext}
\geometry{margin=1in}

% Ustawienia kolorowania składni kodu
\lstset{
  backgroundcolor=\color{darkbackground},
  basicstyle=\ttfamily\color{lighttext},
  keywordstyle=\color{keywordcolor},
  stringstyle=\color{stringcolor},
  commentstyle=\color{commentcolor}\textit,
  numberstyle=\tiny\color{numbercolor},
  identifierstyle=\color{functioncolor},
  breaklines=true,
  frame=single,
  rulecolor=\color{lighttext},
  tabsize=2,
  showstringspaces=false,
  captionpos=b,
}

\title{Zaawansowany Przewodnik po React.js}
\author{Kacper Renkel}
\date{\today}

\begin{document}

\maketitle

\tableofcontents
\newpage

\section{Wprowadzenie}
React to biblioteka JavaScript służąca do tworzenia dynamicznych interfejsów użytkownika. W tym dokumencie omówimy zaawansowane aspekty Reacta, w tym szczegółowy opis hooków, serwisy, interceptory, zarządzanie stanem i routing.

\section{Komponenty}
Komponenty są podstawowymi elementami Reacta. Możemy je tworzyć zarówno jako funkcjonalne komponenty, jak i komponenty klasowe.

\subsection{Funkcyjne komponenty}
Funkcyjne komponenty są prostymi funkcjami, które przyjmują \texttt{props} i zwracają elementy Reacta.

\begin{lstlisting}[language=JavaScript, caption=Prosty funkcjonalny komponent]
import React from 'react';

function Welcome(props) {
  return <h1>Hello, {props.name}</h1>;
}

export default Welcome;
\end{lstlisting}

\subsection{Komponenty klasowe}
Komponenty klasowe oferują bardziej rozbudowaną funkcjonalność, taką jak zarządzanie stanem.

\begin{lstlisting}[language=JavaScript, caption=Komponent klasowy]
import React, { Component } from 'react';

class Welcome extends Component {
  render() {
    return <h1>Hello, {this.props.name}</h1>;
  }
}

export default Welcome;
\end{lstlisting}

\section{Hooki w React}
Hooki to funkcje, które pozwalają na "zaczepienie" się w wewnętrzne mechanizmy Reacta, takie jak stan, cykl życia komponentu, itp.

\subsection{\texttt{useState}}
\texttt{useState} to hook, który umożliwia zarządzanie stanem w funkcjonalnych komponentach.

\begin{lstlisting}[language=JavaScript, caption=Przykład użycia \texttt{useState}]
import React, { useState } from 'react';

function Counter() {
  const [count, setCount] = useState(0);

  return (
    <div>
      <p>You clicked {count} times</p>
      <button onClick={() => setCount(count + 1)}>Click me</button>
    </div>
  );
}

export default Counter;
\end{lstlisting}

\subsection{\texttt{useEffect}}
\texttt{useEffect} pozwala na wykonywanie efektów ubocznych, takich jak pobieranie danych z API, po każdym renderowaniu komponentu.

\begin{lstlisting}[language=JavaScript, caption=Przykład użycia \texttt{useEffect}]
import React, { useState, useEffect } from 'react';

function DataFetcher({ url }) {
  const [data, setData] = useState(null);

  useEffect(() => {
    fetch(url)
      .then(response => response.json())
      .then(data => setData(data));
  }, [url]);

  return (
    <div>
      <pre>{JSON.stringify(data, null, 2)}</pre>
    </div>
  );
}

export default DataFetcher;
\end{lstlisting}

\subsection{\texttt{useContext}}
\texttt{useContext} pozwala na korzystanie z kontekstu Reacta w komponentach funkcyjnych.

\begin{lstlisting}[language=JavaScript, caption=Przykład użycia \texttt{useContext}]
import React, { useContext } from 'react';

const ThemeContext = React.createContext('light');

function ThemedButton() {
  const theme = useContext(ThemeContext);
  return <button className={theme}>I am styled by theme context!</button>;
}

export default ThemedButton;
\end{lstlisting}

\subsection{\texttt{useReducer}}
\texttt{useReducer} to zaawansowany hook do zarządzania stanem komponentu, podobny do \texttt{useState}, ale umożliwiający bardziej złożoną logikę aktualizacji stanu.

\begin{lstlisting}[language=JavaScript, caption=Przykład użycia \texttt{useReducer}]
import React, { useReducer } from 'react';

function reducer(state, action) {
  switch (action.type) {
    case 'increment':
      return { count: state.count + 1 };
    case 'decrement':
      return { count: state.count - 1 };
    default:
      throw new Error();
  }
}

function Counter() {
  const [state, dispatch] = useReducer(reducer, { count: 0 });

  return (
    <div>
      <p>Count: {state.count}</p>
      <button onClick={() => dispatch({ type: 'increment' })}>+</button>
      <button onClick={() => dispatch({ type: 'decrement' })}>-</button>
    </div>
  );
}

export default Counter;
\end{lstlisting}

\subsection{\texttt{useRef}}
\texttt{useRef} pozwala na tworzenie referencji do elementów DOM lub przechowywanie zmiennych, które nie powodują ponownego renderowania komponentu przy zmianie wartości.

\begin{lstlisting}[language=JavaScript, caption=Przykład użycia \texttt{useRef}]
import React, { useRef } from 'react';

function FocusableInput() {
  const inputRef = useRef(null);

  const focusInput = () => {
    inputRef.current.focus();
  };

  return (
    <div>
      <input ref={inputRef} type="text" />
      <button onClick={focusInput}>Focus the input</button>
    </div>
  );
}

export default FocusableInput;
\end{lstlisting}

\subsection{\texttt{useMemo}}
\texttt{useMemo} pozwala na memoizację wartości, która zostanie ponownie obliczona tylko wtedy, gdy zależności się zmienią, co jest przydatne do optymalizacji wydajności.

\begin{lstlisting}[language=JavaScript, caption=Przykład użycia \texttt{useMemo}]
import React, { useMemo } from 'react';

function ExpensiveCalculationComponent({ num }) {
  const calculation = useMemo(() => {
    return expensiveCalculation(num);
  }, [num]);

  return <div>Result: {calculation}</div>;
}

function expensiveCalculation(num) {
  // skomplikowana operacja
  return num * 2;
}

export default ExpensiveCalculationComponent;
\end{lstlisting}

\subsection{\texttt{useCallback}}
\texttt{useCallback} memoizuje funkcje, zapobiegając ich ponownemu tworzeniu podczas każdego renderowania, co jest przydatne, gdy przekazujemy funkcje do komponentów zależnych.

\begin{lstlisting}[language=JavaScript, caption=Przykład użycia \texttt{useCallback}]
import React, { useState, useCallback } from 'react';

function ParentComponent() {
  const [count, setCount] = useState(0);

  const increment = useCallback(() => {
    setCount(count + 1);
  }, [count]);

  return (
    <div>
      <p>Count: {count}</p>
      <ChildComponent onIncrement={increment} />
    </div>
  );
}

function ChildComponent({ onIncrement }) {
  return <button onClick={onIncrement}>Increment</button>;
}

export default ParentComponent;
\end{lstlisting}

\section{Zaawansowane zarządzanie stanem}
Poza podstawowymi hookami, takimi jak \texttt{useState} i \texttt{useReducer}, React oferuje narzędzia takie jak Context API oraz biblioteki zewnętrzne, jak Redux, do zarządzania globalnym stanem aplikacji.

\subsection{Context API}
Context API pozwala na przekazywanie danych przez drzewo komponentów bez konieczności ręcznego przekazywania \texttt{props} na każdym poziomie.

\begin{lstlisting}[language=JavaScript, caption=Przykład użycia Context API]
import React, { createContext, useContext, useState } from 'react';

const CountContext = createContext();

function CounterProvider({ children }) {
  const [count, setCount] = useState(0);
  return (
    <CountContext.Provider value={{ count, setCount }}>
      {children}
    </CountContext.Provider>
  );
}

function CounterDisplay() {
  const { count } = useContext(CountContext);
  return <div>Count: {count}</div>;
}

function IncrementButton() {
  const { setCount } = useContext(CountContext);
  return <button onClick={() => setCount(count => count + 1)}>Increment</button>;
}

function App() {
  return (
    <CounterProvider>
      <CounterDisplay />
      <IncrementButton />
    </CounterProvider>
  );
}

export default App;
\end{lstlisting}

\subsection{Redux}
Redux to popularna biblioteka do zarządzania stanem aplikacji. Stan aplikacji jest przechowywany w jednym miejscu zwanym \texttt{store}, a zmiany w stanie są realizowane przez \texttt{actions} i \texttt{reducers}.

\section{Routing}
Routing w React można zrealizować za pomocą biblioteki \texttt{react-router}, która pozwala na tworzenie nawigacji między różnymi widokami aplikacji.

\begin{lstlisting}[language=JavaScript, caption=Przykład użycia \texttt{react-router}]
import React from 'react';
import { BrowserRouter as Router, Route, Switch } from 'react-router-dom';

function Home() {
  return <h2>Home</h2>;
}

function About() {
  return <h2>About</h2>;
}

function App() {
  return (
    <Router>
      <Switch>
        <Route path="/about">
          <About />
        </Route>
        <Route path="/">
          <Home />
        </Route>
      </Switch>
    </Router>
  );
}

export default App;
\end{lstlisting}

\section{Serwisy}
Serwisy w React mogą być tworzone jako osobne moduły do zarządzania logiką aplikacji, szczególnie w zakresie interakcji z API.

\begin{lstlisting}[language=JavaScript, caption=Przykład serwisu API]
class ApiService {
  static async fetchData(url) {
    const response = await fetch(url);
    if (!response.ok) {
      throw new Error('Failed to fetch data');
    }
    return response.json();
  }
}

export default ApiService;
\end{lstlisting}

\section{Interceptors}
Interceptors mogą być używane do przechwytywania żądań lub odpowiedzi HTTP w celu dodania logiki przed lub po wysłaniu żądania.

\begin{lstlisting}[language=JavaScript, caption=Przykład użycia Interceptorów w Axios]
import axios from 'axios';

axios.interceptors.request.use(request => {
  // Dodaj nagłówki, tokeny itp.
  return request;
});

axios.interceptors.response.use(response => {
  // Przetwarzaj odpowiedzi, błędy itp.
  return response;
}, error => {
  // Obsługa błędów
  return Promise.reject(error);
});
\end{lstlisting}

\section{Podsumowanie}
W tym dokumencie omówiliśmy zaawansowane aspekty tworzenia aplikacji w React, takie jak użycie hooków, serwisów, interceptorów oraz zarządzanie stanem. React oferuje elastyczne narzędzia do budowania złożonych interfejsów użytkownika, co czyni go jednym z najpopularniejszych wyborów wśród deweloperów front-endu.

\end{document}
