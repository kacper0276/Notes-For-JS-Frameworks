\documentclass[a4paper,12pt]{article}
\usepackage[utf8]{inputenc}
\usepackage{amsmath}
\usepackage{listings}
\usepackage{xcolor}
\usepackage{hyperref}
\usepackage{geometry}
\usepackage{pagecolor}

% Ustawienia ciemnego motywu
\definecolor{darkbackground}{HTML}{2E3440}
\definecolor{lighttext}{HTML}{D8DEE9}
\definecolor{keywordcolor}{HTML}{81A1C1}
\definecolor{stringcolor}{HTML}{A3BE8C}
\definecolor{commentcolor}{HTML}{616E88}
\definecolor{numbercolor}{HTML}{B48EAD}
\definecolor{functioncolor}{HTML}{88C0D0}

\pagecolor{darkbackground}
\color{lighttext}
\geometry{margin=1in}

% Ustawienia kolorowania składni kodu
\lstset{
  backgroundcolor=\color{darkbackground},
  basicstyle=\ttfamily\color{lighttext},
  keywordstyle=\color{keywordcolor},
  stringstyle=\color{stringcolor},
  commentstyle=\color{commentcolor}\textit,
  numberstyle=\tiny\color{numbercolor},
  identifierstyle=\color{functioncolor},
  breaklines=true,
  frame=single,
  rulecolor=\color{lighttext},
  tabsize=2,
  showstringspaces=false,
  captionpos=b,
}

\title{Zaawansowany Przewodnik po Angularze}
\author{Kacper Renkel}
\date{\today}

\begin{document}

\maketitle

\tableofcontents
\newpage

\section{Wprowadzenie}
Angular to platforma i framework do budowania złożonych aplikacji webowych. W tym dokumencie omówimy zaawansowane aspekty Angulara, takie jak Guardy, Interceptory, Serwisy, Moduły, Routing, oraz inne zaawansowane funkcje.

\section{Struktura aplikacji Angular}
Angular opiera się na modularnej architekturze, co pozwala na łatwe zarządzanie złożonymi aplikacjami poprzez podział na moduły.

\subsection{Moduły}
Moduły w Angularze to zbiory komponentów, serwisów, dyrektyw i innych zasobów, które są grupowane razem w celu ułatwienia zarządzania i organizacji kodu.

\begin{lstlisting}[language=TypeScript, caption=Definicja modułu w Angularze]
import { NgModule } from '@angular/core';
import { BrowserModule } from '@angular/platform-browser';
import { AppComponent } from './app.component';
import { SharedModule } from './shared/shared.module';

@NgModule({
  declarations: [
    AppComponent
  ],
  imports: [
    BrowserModule,
    SharedModule
  ],
  providers: [],
  bootstrap: [AppComponent]
})
export class AppModule { }
\end{lstlisting}

\section{Komponenty}
Komponenty są podstawowymi elementami aplikacji Angular, które odpowiadają za interfejs użytkownika.

\begin{lstlisting}[language=TypeScript, caption=Przykładowy komponent]
import { Component } from '@angular/core';

@Component({
  selector: 'app-hello-world',
  template: '<h1>Hello, World!</h1>',
  styleUrls: ['./hello-world.component.css']
})
export class HelloWorldComponent { }
\end{lstlisting}

\section{Serwisy}
Serwisy w Angularze to klasy, które zawierają logikę biznesową i są wstrzykiwane do komponentów lub innych serwisów za pomocą mechanizmu Dependency Injection.

\begin{lstlisting}[language=TypeScript, caption=Przykładowy serwis w Angularze]
import { Injectable } from '@angular/core';

@Injectable({
  providedIn: 'root',
})
export class DataService {
  private data: string[] = [];

  addData(item: string) {
    this.data.push(item);
  }

  getData(): string[] {
    return this.data;
  }
}
\end{lstlisting}

\section{Routing}
Routing w Angularze umożliwia nawigację między różnymi widokami aplikacji. Angular Router to elastyczne narzędzie pozwalające na definiowanie tras, ochronę tras za pomocą Guardów oraz ładowanie modułów na żądanie (lazy loading).

\subsection{Definiowanie tras}
Trasy definiuje się w pliku modułu routingowego, gdzie każda trasa jest przypisana do określonego komponentu.

\begin{lstlisting}[language=TypeScript, caption=Przykład konfiguracji tras]
import { NgModule } from '@angular/core';
import { RouterModule, Routes } from '@angular/router';
import { HomeComponent } from './home/home.component';
import { AboutComponent } from './about/about.component';

const routes: Routes = [
  { path: '', component: HomeComponent },
  { path: 'about', component: AboutComponent },
];

@NgModule({
  imports: [RouterModule.forRoot(routes)],
  exports: [RouterModule]
})
export class AppRoutingModule { }
\end{lstlisting}

\subsection{Lazy Loading}
Lazy Loading pozwala na ładowanie modułów dopiero wtedy, gdy są one potrzebne, co znacząco poprawia wydajność aplikacji.

\begin{lstlisting}[language=TypeScript, caption=Przykład Lazy Loading w Angularze]
const routes: Routes = [
  { path: '', component: HomeComponent },
  { 
    path: 'admin', 
    loadChildren: () => import('./admin/admin.module').then(m => m.AdminModule) 
  },
];
\end{lstlisting}

\section{Guardy}
Guardy w Angularze pozwalają na kontrolowanie dostępu do określonych tras na podstawie warunków logicznych. Angular oferuje kilka typów Guardów:

\begin{itemize}
    \item \texttt{CanActivate} – sprawdza, czy użytkownik może przejść do określonej trasy.
    \item \texttt{CanDeactivate} – sprawdza, czy użytkownik może opuścić określoną trasę.
    \item \texttt{CanLoad} – sprawdza, czy moduł może być załadowany na żądanie.
\end{itemize}

\subsection{Implementacja CanActivate}
\begin{lstlisting}[language=TypeScript, caption=Przykład implementacji \texttt{CanActivate}]
import { Injectable } from '@angular/core';
import { CanActivate, Router } from '@angular/router';
import { AuthService } from './auth.service';

@Injectable({
  providedIn: 'root'
})
export class AuthGuard implements CanActivate {

  constructor(private authService: AuthService, private router: Router) {}

  canActivate(): boolean {
    if (this.authService.isAuthenticated()) {
      return true;
    } else {
      this.router.navigate(['/login']);
      return false;
    }
  }
}
\end{lstlisting}

\section{Interceptory}
Interceptory w Angularze są częścią mechanizmu HTTP Client i umożliwiają przechwytywanie oraz modyfikację żądań i odpowiedzi HTTP.

\subsection{Tworzenie Interceptora}
\begin{lstlisting}[language=TypeScript, caption=Przykład implementacji Interceptora]
import { Injectable } from '@angular/core';
import { HttpInterceptor, HttpRequest, HttpHandler, HttpEvent } from '@angular/common/http';
import { Observable } from 'rxjs';

@Injectable()
export class AuthInterceptor implements HttpInterceptor {

  intercept(req: HttpRequest<any>, next: HttpHandler): Observable<HttpEvent<any>> {
    const authToken = 'my-auth-token';
    const authReq = req.clone({
      setHeaders: { Authorization: `Bearer ${authToken}` }
    });
    return next.handle(authReq);
  }
}
\end{lstlisting}

\subsection{Rejestrowanie Interceptora}
Aby zarejestrować Interceptor, należy dodać go do tablicy \texttt{providers} w module aplikacji.

\begin{lstlisting}[language=TypeScript, caption=Rejestracja Interceptora]
import { HTTP_INTERCEPTORS } from '@angular/common/http';
import { AuthInterceptor } from './auth.interceptor';

@NgModule({
  providers: [
    { provide: HTTP_INTERCEPTORS, useClass: AuthInterceptor, multi: true },
  ],
})
export class AppModule { }
\end{lstlisting}

\section{Formularze}
Angular oferuje dwa podejścia do tworzenia formularzy: Template-Driven i Reactive Forms. Reactive Forms dają większą kontrolę nad walidacją i stanem formularza.

\subsection{Reactive Forms}
Reactive Forms pozwalają na bardziej dynamiczne zarządzanie formularzami poprzez zdefiniowanie formy w TypeScript zamiast HTML.

\begin{lstlisting}[language=TypeScript, caption=Przykład Reactive Form]
import { Component } from '@angular/core';
import { FormGroup, FormControl, Validators } from '@angular/forms';

@Component({
  selector: 'app-registration',
  templateUrl: './registration.component.html'
})
export class RegistrationComponent {
  registrationForm = new FormGroup({
    username: new FormControl('', Validators.required),
    email: new FormControl('', [Validators.required, Validators.email]),
    password: new FormControl('', [Validators.required, Validators.minLength(6)]),
  });

  onSubmit() {
    console.log(this.registrationForm.value);
  }
}
\end{lstlisting}

\section{HttpClient}
\texttt{HttpClient} w Angularze jest narzędziem do komunikacji z serwerami za pomocą żądań HTTP. Umożliwia obsługę różnych metod HTTP, takich jak \texttt{GET}, \texttt{POST}, \texttt{PUT}, \texttt{DELETE}.

\begin{lstlisting}[language=TypeScript, caption=Przykład użycia HttpClient w Angularze]
import { Injectable } from '@angular/core';
import { HttpClient } from '@angular/common/http';
import { Observable } from 'rxjs';

@Injectable({
  providedIn: 'root'
})
export class ApiService {

  constructor(private http: HttpClient) { }

  getData(): Observable<any> {
    return this.http.get('https://api.example.com/data');
  }

  postData(data: any): Observable<any> {
    return this.http.post('https://api.example.com/data', data);
  }
}
\end{lstlisting}

\section{Middleware w Angularze}
Middleware w Angularze może być zaimplementowany jako Interceptor lub jako mechanizm Guard. Oba podejścia pozwalają na przechwytywanie i manipulowanie żądaniami HTTP lub nawigacją.

\section{JWT (JSON Web Token)}
JWT jest standardem używanym do bezpiecznego przekazywania informacji między serwerem a klientem. W Angularze JWT można łatwo zintegrować z aplikacją przy użyciu Interceptorów do automatycznego dołączania tokenów do nagłówków żądań.

\subsection{Implementacja JWT w Angularze}
Aby zintegrować JWT w Angularze, należy stworzyć serwis do obsługi tokenów oraz Interceptor do automatycznego dołączania tokenu do żądań HTTP.

\begin{lstlisting}[language=TypeScript, caption=Przykład serwisu do obsługi JWT]
import { Injectable } from '@angular/core';

@Injectable({
  providedIn: 'root',
})
export class JwtService {
  private readonly TOKEN_KEY = 'auth-token';

  saveToken(token: string) {
    localStorage.setItem(this.TOKEN_KEY, token);
  }

  getToken(): string | null {
    return localStorage.getItem(this.TOKEN_KEY);
  }

  clearToken() {
    localStorage.removeItem(this.TOKEN_KEY);
  }
}
\end{lstlisting}

\section{Podsumowanie}
Angular jest potężnym narzędziem do tworzenia złożonych aplikacji webowych. Dzięki zaawansowanym funkcjom takim jak Guardy, Interceptory, Lazy Loading, oraz rozbudowanemu systemowi modułów, Angular umożliwia tworzenie skalowalnych i dobrze zorganizowanych aplikacji. Dokument ten przedstawia kluczowe aspekty Angulara, które są niezbędne dla każdego dewelopera pragnącego tworzyć nowoczesne i zaawansowane aplikacje.

\end{document}
