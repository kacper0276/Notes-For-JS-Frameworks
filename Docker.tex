\documentclass[a4paper,12pt]{article}
\usepackage{graphicx}
\usepackage{listings}
\usepackage{xcolor}
\usepackage{hyperref}
\usepackage{geometry}

\geometry{a4paper, margin=1in}

\title{Docker: Przewodnik po Podstawowych Poleceniach i Integracji z Frameworkami}
\author{Kacper Renkel}
\date{\today}

\begin{document}

\maketitle
\tableofcontents

\newpage

\section{Wprowadzenie do Dockera}
Docker to platforma, która umożliwia deweloperom i administratorom systemów tworzenie, wdrażanie i uruchamianie aplikacji w kontenerach. Kontenery pozwalają na spójne środowisko uruchomieniowe niezależnie od infrastruktury.

\section{Podstawowe Polecenia Dockera}

\subsection{Zarządzanie Obrazami}
\begin{itemize}
    \item \textbf{docker pull} - Pobieranie obrazu z Docker Hub.
    \item \textbf{docker images} - Wyświetlanie listy lokalnie dostępnych obrazów.
    \item \textbf{docker rmi} - Usuwanie obrazu.
\end{itemize}

\subsection{Zarządzanie Kontenerami}
\begin{itemize}
    \item \textbf{docker run} - Tworzenie i uruchamianie nowego kontenera.
    \item \textbf{docker ps} - Wyświetlanie uruchomionych kontenerów.
    \item \textbf{docker stop} - Zatrzymywanie kontenera.
    \item \textbf{docker rm} - Usuwanie zatrzymanego kontenera.
\end{itemize}

\subsection{Zarządzanie Sieciami}
\begin{itemize}
    \item \textbf{docker network create} - Tworzenie nowej sieci.
    \item \textbf{docker network ls} - Wyświetlanie listy sieci.
    \item \textbf{docker network inspect} - Wyświetlanie szczegółów sieci.
    \item \textbf{docker network rm} - Usuwanie sieci.
\end{itemize}

\subsection{Zarządzanie Woluminami}
\begin{itemize}
    \item \textbf{docker volume create} - Tworzenie nowego wolumenu.
    \item \textbf{docker volume ls} - Wyświetlanie listy woluminów.
    \item \textbf{docker volume inspect} - Wyświetlanie szczegółów wolumenu.
    \item \textbf{docker volume rm} - Usuwanie wolumenu.
\end{itemize}

\section{Zarządzanie Sieciami w Dockerze}
Docker umożliwia tworzenie i zarządzanie różnymi typami sieci, w tym bridge, host i overlay.

\subsection{Tworzenie Sieci}
Polecenie \texttt{docker network create} służy do tworzenia nowej sieci. Na przykład:
\begin{lstlisting}[language=bash]
docker network create my_network
\end{lstlisting}

\subsection{Inspekcja Sieci}
Aby zobaczyć szczegóły sieci:
\begin{lstlisting}[language=bash]
docker network inspect my_network
\end{lstlisting}

\subsection{Usuwanie Sieci}
Usuwanie sieci za pomocą:
\begin{lstlisting}[language=bash]
docker network rm my_network
\end{lstlisting}

\section{Uruchamianie Terminala Kontenera}
Aby uruchomić interaktywną sesję w kontenerze:
\begin{lstlisting}[language=bash]
docker exec -it <container_id> /bin/bash
\end{lstlisting}

\section{Polecenia używane w Dockerfile}

Dockerfile jest plikiem tekstowym, który zawiera wszystkie instrukcje potrzebne do zbudowania obrazu Dockera.

\begin{itemize}
    \item \textbf{FROM} - Określa bazowy obraz, na którym będzie budowany nowy obraz. Może być z Docker Hub lub z innego rejestru.
    \item \textbf{RUN} - Uruchamia komendy w czasie budowania obrazu. Służy do instalowania zależności i konfiguracji systemu.
    \item \textbf{CMD} - Definiuje domyślne polecenie do uruchomienia po starcie kontenera. Może być nadpisane przez \texttt{docker run}.
    \item \textbf{EXPOSE} - Deklaruje otwarty port, na którym aplikacja będzie dostępna.
    \item \textbf{ENV} - Ustawia zmienne środowiskowe wewnątrz obrazu.
    \item \textbf{COPY} i \textbf{ADD} - Kopiuje pliki do obrazu. \textbf{COPY} służy do kopiowania plików z hosta, a \textbf{ADD} obsługuje także pobieranie z URL.
    \item \textbf{WORKDIR} - Ustawia katalog roboczy kontenera, w którym będą wykonywane wszystkie następne komendy.
    \item \textbf{ENTRYPOINT} - Ustawia domyślne polecenie, które nie może być nadpisane. Często używane z \textbf{CMD}.
    \item \textbf{ARG} - Definiuje zmienne używane tylko w trakcie budowania obrazu.
    \item \textbf{VOLUME} - Definiuje punkt montowania woluminu.
    \item \textbf{LABEL} - Służy do dodawania metadanych do obrazu, takich jak wersje, informacje o autorze.
    \item \textbf{USER} - Określa użytkownika, w kontekście którego będą wykonywane polecenia w kontenerze.
    \item \textbf{ONBUILD} - Definiuje instrukcje, które mają być uruchamiane w czasie budowy obrazu pochodnego.
\end{itemize}

\section{Polecenia używane w docker-compose}

`docker-compose.yml` jest plikiem definiującym konfigurację wielu kontenerów dla aplikacji wielousługowych.

\begin{itemize}
    \item \textbf{version} - Wersja używanego API Compose, np. '3.8'.
    \item \textbf{services} - Definiuje listę usług (kontenerów) w aplikacji.
    \item \textbf{image} - Określa obraz do uruchomienia kontenera.
    \item \textbf{build} - Budowanie obrazów z Dockerfile.
    \item \textbf{volumes} - Montowanie woluminów. Pozwala na współdzielenie plików między hostem a kontenerem.
    \item \textbf{networks} - Konfiguracja sieci między kontenerami.
    \item \textbf{environment} - Ustawianie zmiennych środowiskowych w kontenerze.
    \item \textbf{ports} - Mapowanie portów między hostem a kontenerem.
    \item \textbf{depends\_on} - Określa kolejność uruchamiania kontenerów.
    \item \textbf{command} - Zmienia domyślne polecenie uruchamiane w kontenerze.
    \item \textbf{restart} - Definiuje politykę restartu kontenera, np. \texttt{always}, \texttt{on-failure}.
    \item \textbf{links} - Tworzenie połączeń między kontenerami.
    \item \textbf{extra\_hosts} - Dodawanie wpisów do pliku /etc/hosts w kontenerze.
    \item \textbf{secrets} - Ustawia tajemnice używane przez kontenery, np. hasła.
    \item \textbf{configs} - Zarządza konfiguracjami do użycia w kontenerach.
\end{itemize}

\section{Automatyczne przeładowanie kontenera po zmianach w plikach}

Aby osiągnąć automatyczne przeładowanie kontenerów w Dockerze po zmianach w plikach, można skorzystać z mechanizmu woluminów Dockera. Woluminy pozwalają na synchronizację plików między systemem plików hosta a kontenerem, dzięki czemu zmiany w kodzie źródłowym na hoście są od razu odzwierciedlane w kontenerze. Dzięki temu, po odpowiednim skonfigurowaniu kontenerów, zmiany w plikach będą automatycznie przeładowywać aplikację wewnątrz kontenera.

\section{Przykłady konfiguracji dla różnych frameworków}

Poniżej znajdują się przykłady konfiguracji plików Dockerfile i docker-compose.yml dla różnych popularnych frameworków i języków.

\subsection{Node.js i NestJS}

Dockerfile:
\begin{lstlisting}[language=Dockerfile]
FROM node:14
WORKDIR /app
COPY package*.json ./
RUN npm install
COPY . .
EXPOSE 3000
CMD ["node", "app.js"]
\end{lstlisting}

docker-compose.yml:
\begin{lstlisting}[language=yaml]
version: '3'
services:
  node-app:
    build: .
    ports:
      - "3000:3000"
    volumes:
      - .:/app
\end{lstlisting}

\subsection{Angular, Vue.js, React}

Dockerfile dla Angulara:
\begin{lstlisting}[language=Dockerfile]
FROM node:14
WORKDIR /app
COPY package*.json ./
RUN npm install
COPY . .
EXPOSE 4200
CMD ["npm", "start"]
\end{lstlisting}

docker-compose.yml:
\begin{lstlisting}[language=yaml]
version: '3'
services:
  frontend:
    build: .
    ports:
      - "4200:4200"
    volumes:
      - .:/app
\end{lstlisting}

\subsection{Spring Boot}

Dockerfile:
\begin{lstlisting}[language=Dockerfile]
FROM openjdk:11
WORKDIR /app
COPY . .
RUN ./mvnw package
EXPOSE 8080
CMD ["java", "-jar", "target/myapp.jar"]
\end{lstlisting}

docker-compose.yml:
\begin{lstlisting}[language=yaml]
version: '3'
services:
  spring-app:
    build: .
    ports:
      - "8080:8080"
    volumes:
      - .:/app
\end{lstlisting}

\subsection{.NET Core}

Dockerfile:
\begin{lstlisting}[language=Dockerfile]
FROM mcr.microsoft.com/dotnet/aspnet:5.0 AS base
WORKDIR /app
COPY . .
EXPOSE 5000
CMD ["dotnet", "myapp.dll"]
\end{lstlisting}

docker-compose.yml:
\begin{lstlisting}[language=yaml]
version: '3'
services:
  dotnet-app:
    build: .
    ports:
      - "5000:5000"
    volumes:
      - .:/app
\end{lstlisting}

\section{Podsumowanie}
Docker jest potężnym narzędziem, które umożliwia łatwe tworzenie, wdrażanie i uruchamianie aplikacji w kontenerach. Dzięki użyciu Dockerfile i docker-compose, można zautomatyzować procesy budowy i uruchamiania aplikacji oraz zarządzać ich środowiskiem. Integracja z popularnymi frameworkami pozwala na łatwe tworzenie przenośnych aplikacji, które mogą być uruchamiane w różnych środowiskach bez zmian w kodzie.
\end{document}
