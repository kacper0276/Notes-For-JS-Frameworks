\documentclass[a4paper,12pt]{article}
\usepackage[utf8]{inputenc}
\usepackage{amsmath}
\usepackage{listings}
\usepackage{xcolor}
\usepackage{hyperref}
\usepackage{geometry}
\usepackage{pagecolor}

% Ustawienia ciemnego motywu
\definecolor{darkbackground}{HTML}{2E3440}
\definecolor{lighttext}{HTML}{D8DEE9}
\definecolor{keywordcolor}{HTML}{81A1C1}
\definecolor{stringcolor}{HTML}{A3BE8C}
\definecolor{commentcolor}{HTML}{616E88}
\definecolor{numbercolor}{HTML}{B48EAD}
\definecolor{functioncolor}{HTML}{88C0D0}

\pagecolor{darkbackground}
\color{lighttext}
\geometry{margin=1in}

% Ustawienia kolorowania składni kodu
\lstset{
  backgroundcolor=\color{darkbackground},
  basicstyle=\ttfamily\color{lighttext},
  keywordstyle=\color{keywordcolor},
  stringstyle=\color{stringcolor},
  commentstyle=\color{commentcolor}\textit,
  numberstyle=\tiny\color{numbercolor},
  identifierstyle=\color{functioncolor},
  breaklines=true,
  frame=single,
  rulecolor=\color{lighttext},
  tabsize=2,
  showstringspaces=false,
  captionpos=b,
}

\title{Zaawansowany Przewodnik po NestJS}
\author{Kacper Renkel}
\date{\today}

\begin{document}

\maketitle

\tableofcontents
\newpage

\section{Wprowadzenie}
NestJS to framework do budowania aplikacji serwerowych w Node.js, który wspiera TypeScript i korzysta z dekoratorów oraz architektury opartej na modułach. Ułatwia tworzenie skalowalnych aplikacji dzięki wbudowanym mechanizmom, takim jak Dependency Injection i obsługa middleware.

\section{Kontrolery}
Kontrolery w NestJS są odpowiedzialne za obsługę żądań HTTP i delegowanie zadań do serwisów. Każdy kontroler odpowiada za jedną grupę funkcjonalności aplikacji.

\subsection{Przykład Kontrolera}
\begin{lstlisting}[language=TypeScript, caption=Przykład kontrolera w NestJS]
import { Controller, Get, Post, Body, Param } from '@nestjs/common';
import { UsersService } from './users.service';
import { CreateUserDto } from './dto/create-user.dto';

@Controller('users')
export class UsersController {
  constructor(private readonly usersService: UsersService) {}

  @Get(':id')
  async getUser(@Param('id') id: string) {
    return this.usersService.getUser(id);
  }

  @Post()
  async createUser(@Body() createUserDto: CreateUserDto) {
    return this.usersService.createUser(createUserDto);
  }
}
\end{lstlisting}

\section{Serwisy}
Serwisy w NestJS są klasami, które zawierają logikę biznesową aplikacji i mogą komunikować się z bazą danych lub innymi źródłami danych. Serwisy są wstrzykiwane do kontrolerów i innych serwisów za pomocą Dependency Injection.

\subsection{Przykład Serwisu}
\begin{lstlisting}[language=TypeScript, caption=Przykład serwisu w NestJS]
import { Injectable } from '@nestjs/common';
import { InjectRepository } from '@nestjs/typeorm';
import { Repository } from 'typeorm';
import { User } from './user.entity';
import { CreateUserDto } from './dto/create-user.dto';

@Injectable()
export class UsersService {
  constructor(
    @InjectRepository(User)
    private readonly userRepository: Repository<User>,
  ) {}

  async getUser(id: string): Promise<User> {
    return this.userRepository.findOneBy({ id });
  }

  async createUser(createUserDto: CreateUserDto): Promise<User> {
    const user = this.userRepository.create(createUserDto);
    return this.userRepository.save(user);
  }
}
\end{lstlisting}

\section{Repozytoria}
Repozytoria w NestJS zarządzają interakcją z bazą danych i są zwykle tworzone przy użyciu ORM, takiego jak TypeORM.

\subsection{Przykład Repozytorium}
\begin{lstlisting}[language=TypeScript, caption=Przykład repozytorium w NestJS]
import { EntityRepository, Repository } from 'typeorm';
import { User } from './user.entity';

@EntityRepository(User)
export class UserRepository extends Repository<User> {
  // Dodatkowe metody repozytorium mogą być zdefiniowane tutaj
}
\end{lstlisting}

\section{JWT (JSON Web Token)}
JWT jest używany w NestJS do autoryzacji i uwierzytelniania. Używa się biblioteki `@nestjs/jwt` do generowania i weryfikowania tokenów.

\subsection{Przykład Generowania i Weryfikowania Tokenu}
\begin{lstlisting}[language=TypeScript, caption=Przykład użycia JWT w NestJS]
import { Injectable } from '@nestjs/common';
import { JwtService } from '@nestjs/jwt';

@Injectable()
export class AuthService {
  constructor(private readonly jwtService: JwtService) {}

  async generateToken(payload: any) {
    return this.jwtService.sign(payload);
  }

  async verifyToken(token: string) {
    return this.jwtService.verify(token);
  }
}
\end{lstlisting}

\section{Hashowanie Haseł}
Do hashowania haseł w NestJS można użyć biblioteki `bcrypt`.

\subsection{Przykład Hashowania Haseł}
\begin{lstlisting}[language=TypeScript, caption=Przykład hashowania haseł przy użyciu bcrypt]
import * as bcrypt from 'bcrypt';

export class AuthService {
  async hashPassword(password: string): Promise<string> {
    const salt = await bcrypt.genSalt();
    return bcrypt.hash(password, salt);
  }

  async comparePassword(password: string, hash: string): Promise<boolean> {
    return bcrypt.compare(password, hash);
  }
}
\end{lstlisting}

\section{Middleware}
Middleware w NestJS to klasy implementujące interfejs `NestMiddleware`, które mogą modyfikować żądania i odpowiedzi.

\subsection{Przykład Middleware}
\begin{lstlisting}[language=TypeScript, caption=Przykład middleware w NestJS]
import { Injectable, NestMiddleware } from '@nestjs/common';
import { Request, Response, NextFunction } from 'express';

@Injectable()
export class LoggerMiddleware implements NestMiddleware {
  use(req: Request, res: Response, next: NextFunction) {
    console.log(`Request made to: ${req.url}`);
    next();
  }
}
\end{lstlisting}

\section{Guardy}
Guardy w NestJS służą do autoryzacji i decydują, czy żądanie może przejść do kontrolera. Implementują interfejs `CanActivate`.

\subsection{Przykład Guarcu}
\begin{lstlisting}[language=TypeScript, caption=Przykład guardu w NestJS]
import { Injectable, CanActivate, ExecutionContext } from '@nestjs/common';

@Injectable()
export class AuthGuard implements CanActivate {
  canActivate(context: ExecutionContext): boolean {
    const request = context.switchToHttp().getRequest();
    const token = request.headers['authorization'];
    return token === 'valid-token';
  }
}
\end{lstlisting}

\section{Interceptory}
Interceptory w NestJS mogą modyfikować żądania i odpowiedzi oraz realizować dodatkowe operacje, takie jak logowanie czy modyfikacja danych odpowiedzi.

\subsection{Przykład Interceptora}
\begin{lstlisting}[language=TypeScript, caption=Przykład interceptora w NestJS]
import { Injectable, NestInterceptor, ExecutionContext, CallHandler } from '@nestjs/common';
import { Observable } from 'rxjs';
import { tap } from 'rxjs/operators';

@Injectable()
export class ResponseInterceptor implements NestInterceptor {
  intercept(context: ExecutionContext, next: CallHandler): Observable<any> {
    console.log('Before handling request');
    return next.handle().pipe(
      tap(() => console.log('After handling request'))
    );
  }
}
\end{lstlisting}

\section{Protected Routes}
Protected routes w NestJS są zabezpieczane za pomocą guardów, które sprawdzają autoryzację przed dostępem do chronionych zasobów.

\subsection{Przykład Chronionej Trasy}
\begin{lstlisting}[language=TypeScript, caption=Przykład chronionej trasy w NestJS]
import { Module } from '@nestjs/common';
import { AuthGuard } from './auth.guard';
import { UsersController } from './users.controller';

@Module({
  controllers: [UsersController],
  providers: [AuthGuard],
})
export class AppModule {}
\end{lstlisting}

\section{Łączenie się z Bazą Danych}
NestJS wspiera wiele bibliotek ORM i ODM, takich jak TypeORM i Mongoose.

\subsection{TypeORM}
TypeORM to popularny ORM dla TypeScript i JavaScript. Aby połączyć się z bazą danych przy użyciu TypeORM, należy skonfigurować `TypeOrmModule` w module głównym.

\subsubsection{Przykład Konfiguracji TypeORM}
\begin{lstlisting}[language=TypeScript, caption=Konfiguracja TypeORM w NestJS]
import { TypeOrmModule } from '@nestjs/typeorm';
import { User } from './user.entity';

@Module({
  imports: [
    TypeOrmModule.forRoot({
      type: 'mysql',
      host: 'localhost',
      port: 3306,
      username: 'root',
      password: 'password',
      database: 'test',
      entities: [User],
      synchronize: true,
    }),
    TypeOrmModule.forFeature([User]),
  ],
  // inne konfiguracje
})
export class AppModule {}
\end{lstlisting}

\subsection{Mongoose}
Mongoose to popularne ODM dla MongoDB. Aby połączyć się z MongoDB przy użyciu Mongoose, należy skonfigurować `MongooseModule` w module głównym.

\subsubsection{Przykład Konfiguracji Mongoose}
\begin{lstlisting}[language=TypeScript, caption=Konfiguracja Mongoose w NestJS]
import { MongooseModule } from '@nestjs/mongoose';
import { User, UserSchema } from './schemas/user.schema';

@Module({
  imports: [
    MongooseModule.forRoot('mongodb://localhost/nest'),
    MongooseModule.forFeature([{ name: User.name, schema: UserSchema }]),
  ],
  // inne konfiguracje
})
export class AppModule {}
\end{lstlisting}

\section{Inne Ważne Aspekty}
\subsection{Konfiguracja Bazy Danych}
NestJS obsługuje różne bazy danych przez TypeORM, Mongoose i inne ORM/ODM. Ważne jest, aby dobrze skonfigurować połączenie z bazą danych i zarządzać migracjami oraz schematami.

\subsection{Wstrzykiwanie Zależności}
NestJS używa Dependency Injection (DI) do zarządzania zależnościami, co ułatwia testowanie i modularność aplikacji. DI jest kluczowym elementem architektury NestJS i umożliwia tworzenie luźno powiązanych komponentów.

\end{document}
