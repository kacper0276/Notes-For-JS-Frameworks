\documentclass[a4paper,12pt]{article}
\usepackage[utf8]{inputenc}
\usepackage{amsmath}
\usepackage{listings}
\usepackage{xcolor}
\usepackage{hyperref}
\usepackage{geometry}
\usepackage{pagecolor} % Pakiet do zmiany koloru tła

% Ustawienia ciemnego motywu
\definecolor{darkbackground}{HTML}{2E3440}
\definecolor{lighttext}{HTML}{D8DEE9}
\definecolor{keywordcolor}{HTML}{81A1C1}
\definecolor{stringcolor}{HTML}{A3BE8C}
\definecolor{commentcolor}{HTML}{616E88}
\definecolor{numbercolor}{HTML}{B48EAD}
\definecolor{functioncolor}{HTML}{88C0D0}

\pagecolor{darkbackground}
\color{lighttext}
\geometry{margin=1in}

% Ustawienia kolorowania składni kodu
\lstset{
  backgroundcolor=\color{darkbackground},
  basicstyle=\ttfamily\color{lighttext},
  keywordstyle=\color{keywordcolor},
  stringstyle=\color{stringcolor},
  commentstyle=\color{commentcolor}\textit,
  numberstyle=\tiny\color{numbercolor},
  identifierstyle=\color{functioncolor},
  breaklines=true,
  frame=single,
  rulecolor=\color{lighttext},
  tabsize=2,
  showstringspaces=false,
  captionpos=b,
}

\title{Przewodnik po Vue.js}
\author{Kacper Renkel}
\date{\today}

\begin{document}

\maketitle

\tableofcontents
\newpage

\section{Wprowadzenie}
Vue.js to progresywny framework JavaScriptowy do budowania interfejsów użytkownika. Zapewnia elastyczność oraz możliwość integracji z różnymi projektami, co czyni go popularnym wyborem wśród deweloperów.

\section{Cykl życia komponentów}
Cykl życia komponentu składa się z kilku etapów:
\begin{itemize}
    \item \textbf{beforeCreate} - Inicjalizacja komponentu, brak dostępu do danych i metod.
    \item \textbf{created} - Dostęp do danych i metod, komponent nie jest jeszcze zamontowany na DOM.
    \item \textbf{beforeMount} - Komponent jest gotowy do zamontowania na DOM.
    \item \textbf{mounted} - Komponent został zamontowany na DOM.
    \item \textbf{beforeUpdate} - Przed aktualizacją komponentu w odpowiedzi na zmiany w danych.
    \item \textbf{updated} - Po aktualizacji komponentu na DOM.
    \item \textbf{beforeUnmount} - Przed usunięciem komponentu z DOM.
    \item \textbf{unmounted} - Komponent został usunięty z DOM.
\end{itemize}

\begin{lstlisting}[language=JavaScript, caption=Przykład użycia cyklu życia komponentu]
export default {
  data() {
    return {
      message: 'Hello Vue!'
    }
  },
  created() {
    console.log('Komponent został utworzony');
  },
  mounted() {
    console.log('Komponent został zamontowany na DOM');
  }
}
\end{lstlisting}

\section{Computed Properties i Watch}
\subsection{Computed Properties}
Computed properties to właściwości, które są obliczane na podstawie innych danych. Vue.js buforuje wyniki obliczeń, więc są one aktualizowane tylko wtedy, gdy zależne dane ulegają zmianie.

\begin{lstlisting}[language=JavaScript, caption=Przykład computed properties]
export default {
  data() {
    return {
      firstName: 'John',
      lastName: 'Doe'
    }
  },
  computed: {
    fullName() {
      return this.firstName + ' ' + this.lastName;
    }
  }
}
\end{lstlisting}

\subsection{Watch}
Watch to mechanizm obserwujący zmiany w danych i uruchamiający określoną funkcję w odpowiedzi na te zmiany. Jest użyteczny, gdy potrzebujemy wykonać operacje asynchroniczne lub kosztowne obliczenia.

\begin{lstlisting}[language=JavaScript, caption=Przykład watch]
export default {
  data() {
    return {
      query: ''
    }
  },
  watch: {
    query(newValue, oldValue) {
      this.fetchResults(newValue);
    }
  },
  methods: {
    fetchResults(query) {
      // Logika do pobierania wyników wyszukiwania
    }
  }
}
\end{lstlisting}

\section{Dyrektywy}
Dyrektywy w Vue.js umożliwiają manipulację DOM poprzez specjalne atrybuty. Poniżej znajdują się przykłady popularnych dyrektyw.

\begin{itemize}
    \item \texttt{v-bind} - Służy do dynamicznego przypisania atrybutów lub właściwości do elementów.
    \item \texttt{v-if} / \texttt{v-else-if} / \texttt{v-else} - Renderowanie warunkowe.
    \item \texttt{v-for} - Iteracja po kolekcji danych.
    \item \texttt{v-on} - Obsługa zdarzeń.
    \item \texttt{v-model} - Dwukierunkowe wiązanie danych.
\end{itemize}

\begin{lstlisting}[language=HTML, caption=Przykład dyrektyw w Vue.js]
<div v-if="isVisible">Widoczne</div>
<ul>
  <li v-for="item in items" :key="item.id">{{ item.name }}</li>
</ul>
<button @click="handleClick">Kliknij mnie</button>
<input v-model="message" placeholder="Wpisz wiadomość">
\end{lstlisting}

\section{Przekazywanie danych między komponentami}
\subsection{Od rodzica do dziecka (Props)}
Dane mogą być przekazywane od komponentu rodzica do dziecka poprzez propsy.

\begin{lstlisting}[language=JavaScript, caption=Przekazywanie danych przez props]
<!-- Komponent rodzic -->
<template>
  <ChildComponent :message="parentMessage" />
</template>

<script>
export default {
  data() {
    return {
      parentMessage: 'Hello from parent'
    }
  }
}
</script>

<!-- Komponent dziecko -->
<template>
  <p>{{ message }}</p>
</template>

<script>
export default {
  props: ['message']
}
</script>
\end{lstlisting}

\subsection{Od dziecka do rodzica (Emitowanie zdarzeń)}
Dziecko może komunikować się z rodzicem poprzez emitowanie zdarzeń.

\begin{lstlisting}[language=JavaScript, caption=Emitowanie zdarzenia przez dziecko]
<!-- Komponent dziecko -->
<template>
  <button @click="sendMessage">Send Message</button>
</template>

<script>
export default {
  methods: {
    sendMessage() {
      this.$emit('messageSent', 'Hello from child');
    }
  }
}
\end{lstlisting}

<!-- Komponent rodzic -->
<template>
  <ChildComponent @messageSent="handleMessage" />
</template>

<script>
export default {
  methods: {
    handleMessage(msg) {
      console.log(msg);
    }
  }
}
</script>
\end{lstlisting}

\section{Sloty}
Sloty umożliwiają przekazywanie zawartości HTML z komponentu rodzica do dziecka, co pozwala na większą elastyczność w tworzeniu komponentów wielokrotnego użytku.

\begin{lstlisting}[language=HTML, caption=Przykład użycia slotów]
<!-- Komponent dziecko -->
<template>
  <div class="card">
    <slot></slot> <!-- Standardowy slot -->
  </div>
</template>

<!-- Komponent rodzic -->
<template>
  <CardComponent>
    <p>Treść przekazana przez slot</p>
  </CardComponent>
</template>
\end{lstlisting}

\section{Mixiny}
Mixiny to mechanizm pozwalający na współdzielenie funkcjonalności pomiędzy różnymi komponentami. Umożliwiają one współdzielenie metod, danych, oraz innych opcji.

\begin{lstlisting}[language=JavaScript, caption=Przykład użycia mixinów]
export const myMixin = {
  data() {
    return {
      mixinData: 'To jest dane z mixinu'
    }
  },
  methods: {
    mixinMethod() {
      console.log('To jest metoda z mixinu');
    }
  }
}

export default {
  mixins: [myMixin],
  mounted() {
    console.log(this.mixinData); // 'To jest dane z mixinu'
    this.mixinMethod(); // 'To jest metoda z mixinu'
  }
}
\end{lstlisting}

\section{Provide/Inject}
Mechanizm \texttt{provide/inject} pozwala na przekazywanie danych z komponentu rodzica do dziecka bez konieczności przekazywania ich przez każdy pośredni komponent (prop drilling).

\begin{lstlisting}[language=JavaScript, caption=Przykład użycia provide/inject]
<!-- Komponent rodzic -->
<template>
  <ChildComponent />
</template>

<script>
export default {
  provide() {
    return {
      message: 'Hello from parent'
    }
  }
}
</script>

<!-- Komponent dziecko -->
<template>
  <p>{{ message }}</p>
</template>

<script>
export default {
  inject: ['message']
}
</script>
\end{lstlisting}

\section{Dynamiczne komponenty}
Vue.js umożliwia dynamiczne ładowanie i renderowanie komponentów w oparciu o zmienne.

\begin{lstlisting}[language=HTML, caption=Przykład dynamicznych komponentów]
<template>
  <component :is="currentComponent"></component>
</template>

<script>
export default {
  data() {
    return {
      currentComponent: 'ComponentA'
    }
  },
  components: {
    ComponentA,
    ComponentB
  }
}
</script>
\end{lstlisting}

\section{\texttt{script setup} vs tradycyjny \texttt{script}}
Vue 3 wprowadził nową składnię \texttt{script setup}, która upraszcza kod komponentu.

\begin{itemize}
    \item \textbf{\texttt{script setup}} - Lepsza wydajność, bardziej zwarta składnia, brak potrzeby jawnej deklaracji zwracanych danych.
    \item \textbf{Tradycyjny \texttt{script}} - Większa elastyczność, szczególnie w skomplikowanych komponentach, lepsza czytelność dla większych zespołów.
\end{itemize}

\begin{lstlisting}[language=HTML, caption=Przykład użycia \texttt{script setup}]
<template>
  <p>{{ message }}</p>
</template>

<script setup>
import { ref } from 'vue'

const message = ref('Hello from script setup')
</script>
\end{lstlisting}

W przypadku prostych komponentów i projektów \texttt{script setup} jest zalecany ze względu na prostotę i wydajność. W bardziej złożonych komponentach, szczególnie gdy potrzebujemy pełnej kontroli nad cyklem życia komponentu, tradycyjny \texttt{script} może być lepszym wyborem.

\section{Podsumowanie}
Vue.js oferuje szeroki wachlarz narzędzi do tworzenia dynamicznych, interaktywnych interfejsów użytkownika. Wybór odpowiednich technik i narzędzi zależy od specyficznych wymagań projektu oraz preferencji zespołu.

\end{document}
