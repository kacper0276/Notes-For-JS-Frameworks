\documentclass[a4paper,12pt]{article}
\usepackage[utf8]{inputenc}
\usepackage{amsmath}
\usepackage{listings}
\usepackage{xcolor}
\usepackage{hyperref}
\usepackage{geometry}
\usepackage{pagecolor}

% Ustawienia ciemnego motywu
\definecolor{darkbackground}{HTML}{2E3440}
\definecolor{lighttext}{HTML}{D8DEE9}
\definecolor{keywordcolor}{HTML}{81A1C1}
\definecolor{stringcolor}{HTML}{A3BE8C}
\definecolor{commentcolor}{HTML}{616E88}
\definecolor{numbercolor}{HTML}{B48EAD}
\definecolor{functioncolor}{HTML}{88C0D0}

\pagecolor{darkbackground}
\color{lighttext}
\geometry{margin=1in}

% Ustawienia kolorowania składni kodu
\lstset{
  backgroundcolor=\color{darkbackground},
  basicstyle=\ttfamily\color{lighttext},
  keywordstyle=\color{keywordcolor},
  stringstyle=\color{stringcolor},
  commentstyle=\color{commentcolor}\textit,
  numberstyle=\tiny\color{numbercolor},
  identifierstyle=\color{functioncolor},
  breaklines=true,
  frame=single,
  rulecolor=\color{lighttext},
  tabsize=2,
  showstringspaces=false,
  captionpos=b,
}

\title{Rozszerzona Dokumentacja Operatorów RxJS}
\author{Kacper Renkel}
\date{\today}

\begin{document}

\maketitle

\tableofcontents
\newpage

\section{Wprowadzenie}
RxJS (Reactive Extensions for JavaScript) to biblioteka umożliwiająca programowanie reaktywne. W tym przewodniku szczegółowo omówione są operatory RxJS, które umożliwiają tworzenie, transformowanie, filtrowanie i zarządzanie strumieniami danych.

\section{Operatory do tworzenia Observable}
\subsection{ajax}
Tworzy \texttt{Observable} z odpowiedzi HTTP.

\begin{lstlisting}[language=JavaScript, caption=Przykład użycia \texttt{ajax}]
import { ajax } from 'rxjs/ajax';

const observable = ajax('https://api.example.com/data');
observable.subscribe(response => console.log(response));
// Output: Response from API
\end{lstlisting}

\subsection{bindCallback}
Tworzy \texttt{Observable} z funkcji opóźnionej (callback-based).

\begin{lstlisting}[language=JavaScript, caption=Przykład użycia \texttt{bindCallback}]
import { bindCallback } from 'rxjs';

const callbackFn = (value, cb) => cb(value);
const observable = bindCallback(callbackFn)('Hello');
observable.subscribe(response => console.log(response));
// Output: Hello
\end{lstlisting}

\subsection{bindNodeCallback}
Tworzy \texttt{Observable} z funkcji węzła (Node.js callback-based).

\begin{lstlisting}[language=JavaScript, caption=Przykład użycia \texttt{bindNodeCallback}]
import { bindNodeCallback } from 'rxjs';
import { readFile } from 'fs';

const readFile$ = bindNodeCallback(readFile);
const observable = readFile$('file.txt', 'utf8');
observable.subscribe(content => console.log(content));
// Output: Content of file.txt
\end{lstlisting}

\subsection{defer}
Tworzy \texttt{Observable} dopiero wtedy, gdy jest subskrybowany.

\begin{lstlisting}[language=JavaScript, caption=Przykład użycia \texttt{defer}]
import { defer, of } from 'rxjs';

const observable = defer(() => of(new Date()));
observable.subscribe(date => console.log(date));
// Output: Current date
\end{lstlisting}

\subsection{empty}
Tworzy \texttt{Observable}, który natychmiast kończy się bez emitowania wartości.

\begin{lstlisting}[language=JavaScript, caption=Przykład użycia \texttt{empty}]
import { empty } from 'rxjs';

const observable = empty();
observable.subscribe({
  next(value) { console.log(value); },
  complete() { console.log('Complete'); }
// Output: Complete
\end{lstlisting}

\subsection{from}
Tworzy \texttt{Observable} z obiektów iterowalnych.

\begin{lstlisting}[language=JavaScript, caption=Przykład użycia \texttt{from}]
import { from } from 'rxjs';

const observable = from([10, 20, 30]);
observable.subscribe(value => console.log(value));
// Output: 10, 20, 30
\end{lstlisting}

\subsection{fromEvent}
Tworzy \texttt{Observable} z wydarzeń DOM.

\begin{lstlisting}[language=JavaScript, caption=Przykład użycia \texttt{fromEvent}]
import { fromEvent } from 'rxjs';

const observable = fromEvent(document, 'click');
observable.subscribe(event => console.log('Clicked:', event));
// Output: Click event details
\end{lstlisting}

\subsection{fromEventPattern}
Tworzy \texttt{Observable} z funkcji, które dodają i usuwają nasłuchiwacze zdarzeń.

\begin{lstlisting}[language=JavaScript, caption=Przykład użycia \texttt{fromEventPattern}]
import { fromEventPattern } from 'rxjs';

const observable = fromEventPattern(
  handler => document.addEventListener('click', handler),
  handler => document.removeEventListener('click', handler)
);
observable.subscribe(event => console.log('Clicked:', event));
// Output: Click event details
\end{lstlisting}

\subsection{generate}
Tworzy \texttt{Observable} na podstawie funkcji generującej wartości.

\begin{lstlisting}[language=JavaScript, caption=Przykład użycia \texttt{generate}]
import { generate } from 'rxjs';

const observable = generate(
  0, // Initial state
  x => x < 5, // Condition
  x => x + 1 // Iteration function
);
observable.subscribe(value => console.log(value));
// Output: 0, 1, 2, 3, 4
\end{lstlisting}

\subsection{interval}
Tworzy \texttt{Observable}, który emituje wartości w równych odstępach czasu.

\begin{lstlisting}[language=JavaScript, caption=Przykład użycia \texttt{interval}]
import { interval } from 'rxjs';

const observable = interval(1000); // Emituje co 1 sekundę
observable.subscribe(value => console.log(value));
// Output: 0, 1, 2, 3, ...
\end{lstlisting}

\subsection{of}
Tworzy \texttt{Observable} z wartości przekazanych jako argumenty.

\begin{lstlisting}[language=JavaScript, caption=Przykład użycia \texttt{of}]
import { of } from 'rxjs';

const observable = of(1, 2, 3, 4, 5);
observable.subscribe(value => console.log(value));
// Output: 1, 2, 3, 4, 5
\end{lstlisting}

\subsection{range}
Tworzy \texttt{Observable} emitujący wartości z określonego zakresu.

\begin{lstlisting}[language=JavaScript, caption=Przykład użycia \texttt{range}]
import { range } from 'rxjs';

const observable = range(1, 5); // Emituje wartości od 1 do 5
observable.subscribe(value => console.log(value));
// Output: 1, 2, 3, 4, 5
\end{lstlisting}

\subsection{throwError}
Tworzy \texttt{Observable}, który natychmiast emituje błąd.

\begin{lstlisting}[language=JavaScript, caption=Przykład użycia \texttt{throwError}]
import { throwError } from 'rxjs';

const observable = throwError(() => new Error('Something went wrong'));
observable.subscribe({
  next(value) { console.log(value); },
  error(err) { console.error('Error: ', err); }
});
// Output: Error: Error: Something went wrong
\end{lstlisting}

\subsection{timer}
Tworzy \texttt{Observable}, który emituje wartość po określonym czasie lub w regularnych odstępach.

\begin{lstlisting}[language=JavaScript, caption=Przykład użycia \texttt{timer}]
import { timer } from 'rxjs';

const observable = timer(2000, 1000); // Pierwsza emisja po 2 sekundach, kolejne co 1 sekundę
observable.subscribe(value => console.log(value));
// Output: 0, 1, 2, 3, ...
\end{lstlisting}

\subsection{iif}
Tworzy \texttt{Observable} na podstawie warunku.

\begin{lstlisting}[language=JavaScript, caption=Przykład użycia \texttt{iif}]
import { iif, of, throwError } from 'rxjs';

const condition = true;
const observable = iif(
  () => condition,
  of('Condition met'),
  throwError(() => new Error('Condition not met'))
);
observable.subscribe({
  next(value) { console.log(value); },
  error(err) { console.error('Error: ', err); }
});
// Output: Condition met
\end{lstlisting}

\section{Operatory łączenia tworzące}
\subsection{combineLatest}
Łączy wartości z wielu \texttt{Observable} i emituje, gdy wszystkie źródła wyemitują co najmniej jedną wartość.

\begin{lstlisting}[language=JavaScript, caption=Przykład użycia \texttt{combineLatest}]
import { combineLatest, of } from 'rxjs';

const observable1 = of('A', 'B');
const observable2 = of(1, 2);

combineLatest([observable1, observable2]).subscribe(values => console.log(values));
// Output: ['B', 1], ['B', 2]
\end{lstlisting}

\subsection{concat}
Łączy wiele \texttt{Observable} w kolejności, emitując wartości z jednego źródła, zanim przejdzie do następnego.

\begin{lstlisting}[language=JavaScript, caption=Przykład użycia \texttt{concat}]
import { concat, of } from 'rxjs';

const observable1 = of(1, 2);
const observable2 = of(3, 4);

concat(observable1, observable2).subscribe(value => console.log(value));
// Output: 1, 2, 3, 4
\end{lstlisting}

\subsection{forkJoin}
Emitowanie ostatnich wartości z wielu źródeł, gdy wszystkie źródła zakończą emisję.

\begin{lstlisting}[language=JavaScript, caption=Przykład użycia \texttt{forkJoin}]
import { forkJoin, of } from 'rxjs';

const observable1 = of('A', 'B');
const observable2 = of(1, 2);

forkJoin([observable1, observable2]).subscribe(values => console.log(values));
// Output: [['B'], [2]]
\end{lstlisting}

\subsection{merge}
Łączy wiele \texttt{Observable} i emituje wartości, gdy tylko jeden z nich emituje coś.

\begin{lstlisting}[language=JavaScript, caption=Przykład użycia \texttt{merge}]
import { merge, of } from 'rxjs';

const observable1 = of(1, 2);
const observable2 = of(3, 4);

merge(observable1, observable2).subscribe(value => console.log(value));
// Output: 1, 2, 3, 4
\end{lstlisting}

\subsection{partition}
Dzieli \texttt{Observable} na dwie na podstawie predykatu.

\begin{lstlisting}[language=JavaScript, caption=Przykład użycia \texttt{partition}]
import { partition, from } from 'rxjs';

const [even$, odd$] = partition(from([1, 2, 3, 4, 5]), value => value % 2 === 0);

even$.subscribe(value => console.log('Even:', value));
// Output: Even: 2, 4

odd$.subscribe(value => console.log('Odd:', value));
// Output: Odd: 1, 3, 5
\end{lstlisting}

\subsection{race}
Emitowanie wartości z \texttt{Observable}, który pierwszy wyemituje wartość.

\begin{lstlisting}[language=JavaScript, caption=Przykład użycia \texttt{race}]
import { race, interval, of } from 'rxjs';

const observable1 = interval(1000);
const observable2 = of('A', 'B', 'C');

race(observable1, observable2).subscribe(value => console.log(value));
// Output: 'A', 'B', 'C'
\end{lstlisting}

\subsection{zip}
Łączy wartości z wielu \texttt{Observable} w pary.

\begin{lstlisting}[language=JavaScript, caption=Przykład użycia \texttt{zip}]
import { zip, of } from 'rxjs';

const observable1 = of(1, 2, 3);
const observable2 = of('A', 'B', 'C');

zip(observable1, observable2).subscribe(values => console.log(values));
// Output: [1, 'A'], [2, 'B'], [3, 'C']
\end{lstlisting}

\section{Operatory transformacji}
\subsection{buffer}
Zbiera wartości w buforze na podstawie innego \texttt{Observable}.

\begin{lstlisting}[language=JavaScript, caption=Przykład użycia \texttt{buffer}]
import { interval, buffer, timer } from 'rxjs';

const observable = interval(500).pipe(
  buffer(timer(2000))
);
observable.subscribe(values => console.log(values));
// Output: [0, 1, 2, 3]
\end{lstlisting}

\subsection{bufferCount}
Zbiera wartości w buforze, gdy osiągnie określoną liczbę wartości.

\begin{lstlisting}[language=JavaScript, caption=Przykład użycia \texttt{bufferCount}]
import { interval, bufferCount } from 'rxjs';

const observable = interval(500).pipe(
  bufferCount(3)
);
observable.subscribe(values => console.log(values));
// Output: [0, 1, 2], [3, 4, 5], ...
\end{lstlisting}

\subsection{bufferTime}
Zbiera wartości w buforze na podstawie czasu.

\begin{lstlisting}[language=JavaScript, caption=Przykład użycia \texttt{bufferTime}]
import { interval, bufferTime } from 'rxjs';

const observable = interval(500).pipe(
  bufferTime(2000)
);
observable.subscribe(values => console.log(values));
// Output: [0, 1, 2], [3, 4, 5], ...
\end{lstlisting}

\subsection{bufferToggle}
Zbiera wartości na podstawie otwierającego i zamykającego \texttt{Observable}.

\begin{lstlisting}[language=JavaScript, caption=Przykład użycia \texttt{bufferToggle}]
import { interval, bufferToggle, timer } from 'rxjs';

const observable = interval(500).pipe(
  bufferToggle(timer(2000), () => timer(2000))
);
observable.subscribe(values => console.log(values));
// Output: [0, 1], [2, 3]
\end{lstlisting}

\subsection{bufferWhen}
Zbiera wartości na podstawie emitowanych wartości innego \texttt{Observable}.

\begin{lstlisting}[language=JavaScript, caption=Przykład użycia \texttt{bufferWhen}]
import { interval, bufferWhen, timer } from 'rxjs';

const observable = interval(500).pipe(
  bufferWhen(() => timer(2000))
);
observable.subscribe(values => console.log(values));
// Output: [0, 1], [2, 3], ...
\end{lstlisting}

\subsection{concatMap}
Mapuje wartości na \texttt{Observable} i łączy je sekwencyjnie.

\begin{lstlisting}[language=JavaScript, caption=Przykład użycia \texttt{concatMap}]
import { from, concatMap, of } from 'rxjs';

const observable = from([1, 2, 3]).pipe(
  concatMap(value => of(value * 2))
);
observable.subscribe(value => console.log(value));
// Output: 2, 4, 6
\end{lstlisting}

\subsection{concatMapTo}
Mapuje wszystkie wartości na \texttt{Observable}.

\begin{lstlisting}[language=JavaScript, caption=Przykład użycia \texttt{concatMapTo}]
import { from, concatMapTo, of } from 'rxjs';

const observable = from([1, 2, 3]).pipe(
  concatMapTo(of('A'))
);
observable.subscribe(value => console.log(value));
// Output: 'A', 'A', 'A'
\end{lstlisting}

\subsection{exhaust}
Emituje wartości tylko wtedy, gdy poprzednia emisja zakończyła się.

\begin{lstlisting}[language=JavaScript, caption=Przykład użycia \texttt{exhaust}]
import { interval, exhaust, take } from 'rxjs';

const observable = interval(500).pipe(
  exhaust(),
  take(3)
);
observable.subscribe(value => console.log(value));
// Output: 0, 1, 2
\end{lstlisting}

\subsection{exhaustMap}
Mapuje wartości na \texttt{Observable} i ignoruje nowe wartości, gdy poprzedni \texttt{Observable} nie zakończył emisji.

\begin{lstlisting}[language=JavaScript, caption=Przykład użycia \texttt{exhaustMap}]
import { interval, exhaustMap, of } from 'rxjs';

const observable = interval(500).pipe(
  exhaustMap(value => of(value * 2))
);
observable.subscribe(value => console.log(value));
// Output: 0, 2, 4, ...
\end{lstlisting}

\subsection{expand}
Rekurencyjnie mapuje wartości na nowe \texttt{Observable}.

\begin{lstlisting}[language=JavaScript, caption=Przykład użycia \texttt{expand}]
import { of, expand, take } from 'rxjs';

const observable = of(1).pipe(
  expand(value => of(value + 1)),
  take(5)
);
observable.subscribe(value => console.log(value));
// Output: 1, 2, 3, 4, 5
\end{lstlisting}

\subsection{groupBy}
Grupuje wartości w \texttt{Observable} na podstawie klucza.

\begin{lstlisting}[language=JavaScript, caption=Przykład użycia \texttt{groupBy}]
import { from, groupBy, mergeMap, toArray } from 'rxjs';

const observable = from([1, 2, 3, 4, 5]).pipe(
  groupBy(value => value % 2),
  mergeMap(group => group.pipe(toArray()))
);
observable.subscribe(values => console.log(values));
// Output: [1, 3, 5], [2, 4]
\end{lstlisting}

\subsection{map}
Mapuje wartości na nowe wartości.

\begin{lstlisting}[language=JavaScript, caption=Przykład użycia \texttt{map}]
import { from, map } from 'rxjs';

const observable = from([1, 2, 3]).pipe(
  map(value => value * 2)
);
observable.subscribe(value => console.log(value));
// Output: 2, 4, 6
\end{lstlisting}

\subsection{mapTo}
Mapuje wszystkie wartości na tę samą wartość.

\begin{lstlisting}[language=JavaScript, caption=Przykład użycia \texttt{mapTo}]
import { from, mapTo } from 'rxjs';

const observable = from([1, 2, 3]).pipe(
  mapTo('A')
);
observable.subscribe(value => console.log(value));
// Output: 'A', 'A', 'A'
\end{lstlisting}

\subsection{mergeMap}
Mapuje wartości na \texttt{Observable} i łączy je równolegle.

\begin{lstlisting}[language=JavaScript, caption=Przykład użycia \texttt{mergeMap}]
import { from, mergeMap, of } from 'rxjs';

const observable = from([1, 2, 3]).pipe(
  mergeMap(value => of(value * 2))
);
observable.subscribe(value => console.log(value));
// Output: 2, 4, 6
\end{lstlisting}

\subsection{mergeMapTo}
Mapuje wszystkie wartości na \texttt{Observable} i łączy je równolegle.

\begin{lstlisting}[language=JavaScript, caption=Przykład użycia \texttt{mergeMapTo}]
import { from, mergeMapTo, of } from 'rxjs';

const observable = from([1, 2, 3]).pipe(
  mergeMapTo(of('A'))
);
observable.subscribe(value => console.log(value));
// Output: 'A', 'A', 'A'
\end{lstlisting}

\subsection{mergeScan}
Podobny do \texttt{scan}, ale łączy wartości równolegle.

\begin{lstlisting}[language=JavaScript, caption=Przykład użycia \texttt{mergeScan}]
import { from, mergeScan, of } from 'rxjs';

const observable = from([1, 2, 3]).pipe(
  mergeScan((acc, value) => of(acc + value), 0)
);
observable.subscribe(value => console.log(value));
// Output: 1, 3, 6
\end{lstlisting}

\subsection{pairwise}
Emitowanie par kolejnych wartości.

\begin{lstlisting}[language=JavaScript, caption=Przykład użycia \texttt{pairwise}]
import { from, pairwise } from 'rxjs';

const observable = from([1, 2, 3]).pipe(
  pairwise()
);
observable.subscribe(values => console.log(values));
// Output: [1, 2], [2, 3]
\end{lstlisting}

\subsection{pluck}
Wydobywa wartości z obiektów na podstawie klucza.

\begin{lstlisting}[language=JavaScript, caption=Przykład użycia \texttt{pluck}]
import { from, pluck } from 'rxjs';

const observable = from([{ a: 1 }, { a: 2 }, { a: 3 }]).pipe(
  pluck('a')
);
observable.subscribe(value => console.log(value));
// Output: 1, 2, 3
\end{lstlisting}

\subsection{scan}
Akumuluje wartości na podstawie funkcji akumulatora.

\begin{lstlisting}[language=JavaScript, caption=Przykład użycia \texttt{scan}]
import { from, scan } from 'rxjs';

const observable = from([1, 2, 3]).pipe(
  scan((acc, value) => acc + value, 0)
);
observable.subscribe(value => console.log(value));
// Output: 1, 3, 6
\end{lstlisting}

\subsection{switchScan}
Podobny do \texttt{scan}, ale z zachowaniem wyłącznie najnowszego \texttt{Observable}.

\begin{lstlisting}[language=JavaScript, caption=Przykład użycia \texttt{switchScan}]
import { from, switchScan, of } from 'rxjs';

const observable = from([1, 2, 3]).pipe(
  switchScan((acc, value) => of(acc + value), 0)
);
observable.subscribe(value => console.log(value));
// Output: 1, 3, 6
\end{lstlisting}

\subsection{switchMap}
Mapuje wartości na \texttt{Observable} i subskrybuje najnowszy.

\begin{lstlisting}[language=JavaScript, caption=Przykład użycia \texttt{switchMap}]
import { from, switchMap, of } from 'rxjs';

const observable = from([1, 2, 3]).pipe(
  switchMap(value => of(value * 2))
);
observable.subscribe(value => console.log(value));
// Output: 2, 4, 6
\end{lstlisting}

\subsection{switchMapTo}
Mapuje wszystkie wartości na ten sam \texttt{Observable} i subskrybuje najnowszy.

\begin{lstlisting}[language=JavaScript, caption=Przykład użycia \texttt{switchMapTo}]
import { from, switchMapTo, of } from 'rxjs';

const observable = from([1, 2, 3]).pipe(
  switchMapTo(of('A'))
);
observable.subscribe(value => console.log(value));
// Output: 'A', 'A', 'A'
\end{lstlisting}

\subsection{window}
Zbiera wartości w oknach na podstawie innego \texttt{Observable}.

\begin{lstlisting}[language=JavaScript, caption=Przykład użycia \texttt{window}]
import { interval, window, timer } from 'rxjs';

const observable = interval(500).pipe(
  window(timer(2000))
);
observable.subscribe(windowed => windowed.subscribe(value => console.log(value)));
// Output: 0, 1, 2, 3
\end{lstlisting}

\subsection{windowCount}
Zbiera wartości w oknach na podstawie liczby wartości.

\begin{lstlisting}[language=JavaScript, caption=Przykład użycia \texttt{windowCount}]
import { interval, windowCount } from 'rxjs';

const observable = interval(500).pipe(
  windowCount(3)
);
observable.subscribe(windowed => windowed.subscribe(value => console.log(value)));
// Output: [0, 1, 2], [3, 4, 5], ...
\end{lstlisting}

\subsection{windowTime}
Zbiera wartości w oknach na podstawie czasu.

\begin{lstlisting}[language=JavaScript, caption=Przykład użycia \texttt{windowTime}]
import { interval, windowTime } from 'rxjs';

const observable = interval(500).pipe(
  windowTime(2000)
);
observable.subscribe(windowed => windowed.subscribe(value => console.log(value)));
// Output: [0, 1], [2, 3], ...
\end{lstlisting}

\subsection{windowToggle}
Zbiera wartości na podstawie otwierającego i zamykającego \texttt{Observable}.

\begin{lstlisting}[language=JavaScript, caption=Przykład użycia \texttt{windowToggle}]
import { interval, windowToggle, timer } from 'rxjs';

const observable = interval(500).pipe(
  windowToggle(timer(2000), () => timer(2000))
);
observable.subscribe(windowed => windowed.subscribe(value => console.log(value)));
// Output: [0, 1], [2, 3]
\end{lstlisting}

\subsection{windowWhen}
Zbiera wartości na podstawie emitowanych wartości innego \texttt{Observable}.

\begin{lstlisting}[language=JavaScript, caption=Przykład użycia \texttt{windowWhen}]
import { interval, windowWhen, timer } from 'rxjs';

const observable = interval(500).pipe(
  windowWhen(() => timer(2000))
);
observable.subscribe(windowed => windowed.subscribe(value => console.log(value)));
// Output: [0, 1], [2, 3], ...
\end{lstlisting}

\section{Operatory podziału}
\subsection{share}
Udostępnia wspólną subskrypcję pomiędzy wieloma subskrybentami.

\begin{lstlisting}[language=JavaScript, caption=Przykład użycia \texttt{share}]
import { interval, share } from 'rxjs';

const observable = interval(1000).pipe(
  share()
);
observable.subscribe(value => console.log('Subscriber 1:', value));
observable.subscribe(value => console.log('Subscriber 2:', value));
// Output: Subscriber 1: 0, 1, 2, ... (same for Subscriber 2)
\end{lstlisting}

\subsection{shareReplay}
Udostępnia wspólną subskrypcję i buforuje ostatnie wartości.

\begin{lstlisting}[language=JavaScript, caption=Przykład użycia \texttt{shareReplay}]
import { interval, shareReplay } from 'rxjs';

const observable = interval(1000).pipe(
  shareReplay(1)
);
observable.subscribe(value => console.log('Subscriber 1:', value));
setTimeout(() => {
  observable.subscribe(value => console.log('Subscriber 2:', value));
}, 2000);
// Output: Subscriber 1: 0, 1, 2, ... (Subscriber 2 will start from the latest value)
\end{lstlisting}

\section{Operatory obsługi błędów}
\subsection{catchError}
Obsługuje błędy i zwraca nowe \texttt{Observable} w przypadku błędu.

\begin{lstlisting}[language=JavaScript, caption=Przykład użycia \texttt{catchError}]
import { of, throwError } from 'rxjs';
import { catchError } from 'rxjs/operators';

const observable = throwError(() => new Error('Something went wrong')).pipe(
  catchError(err => {
    console.error('Caught error:', err);
    return of('Recovered value');
  })
);
observable.subscribe(value => console.log(value));
// Output: Caught error: Error: Something went wrong
// Output: Recovered value
\end{lstlisting}

\subsection{retry}
Ponawia subskrypcję \texttt{Observable} w przypadku błędu.

\begin{lstlisting}[language=JavaScript, caption=Przykład użycia \texttt{retry}]
import { throwError } from 'rxjs';
import { retry } from 'rxjs/operators';

const observable = throwError(() => new Error('Failed')).pipe(
  retry(2)
);
observable.subscribe({
  error(err) { console.error('Retry failed:', err); }
});
// Output: Retry failed: Error: Failed
\end{lstlisting}

\subsection{retryWhen}
Ponawia subskrypcję \texttt{Observable} na podstawie logiki dostarczonej przez inny \texttt{Observable}.

\begin{lstlisting}[language=JavaScript, caption=Przykład użycia \texttt{retryWhen}]
import { throwError, timer } from 'rxjs';
import { retryWhen, mergeMap } from 'rxjs/operators';

const observable = throwError(() => new Error('Failed')).pipe(
  retryWhen(errors => errors.pipe(
    mergeMap((error, index) => {
      if (index < 2) {
        console.log('Retrying...');
        return timer(1000);
      }
      return throwError(() => error);
    })
  ))
);
observable.subscribe({
  error(err) { console.error('Retry failed:', err); }
});
// Output: Retrying... (after 1 second)
// Output: Retrying... (after 1 second)
// Output: Retry failed: Error: Failed
\end{lstlisting}
\section{Operatory narzędziowe}
\subsection{tap}
\texttt{tap} pozwala na wykonanie działania na wartościach przepływających przez strumień bez modyfikacji tych wartości.

\begin{lstlisting}[language=JavaScript, caption=Przykład użycia \texttt{tap}]
import { of, tap } from 'rxjs';

const observable = of(1, 2, 3).pipe(
  tap(value => console.log('Before map:', value)),
  map(value => value * 2),
  tap(value => console.log('After map:', value))
);
observable.subscribe(value => console.log('Final value:', value));
// Output:
// Before map: 1
// After map: 2
// Final value: 2
// Before map: 2
// After map: 4
// Final value: 4
// Before map: 3
// After map: 6
// Final value: 6
\end{lstlisting}

\subsection{delay}
\texttt{delay} opóźnia emisję wartości o określony czas.

\begin{lstlisting}[language=JavaScript, caption=Przykład użycia \texttt{delay}]
import { of, delay } from 'rxjs';

const observable = of('Hello').pipe(
  delay(1000)
);
observable.subscribe(value => console.log(value));
// Output (po 1 sekundzie): Hello
\end{lstlisting}

\subsection{delayWhen}
\texttt{delayWhen} opóźnia emisję wartości na podstawie innego \texttt{Observable}.

\begin{lstlisting}[language=JavaScript, caption=Przykład użycia \texttt{delayWhen}]
import { of, delayWhen, timer } from 'rxjs';

const observable = of('Delayed value').pipe(
  delayWhen(() => timer(2000))
);
observable.subscribe(value => console.log(value));
// Output (po 2 sekundach): Delayed value
\end{lstlisting}

\subsection{timeout}
\texttt{timeout} generuje błąd, jeśli nie nadejdzie żadna wartość w określonym czasie.

\begin{lstlisting}[language=JavaScript, caption=Przykład użycia \texttt{timeout}]
import { of, timeout } from 'rxjs';

const observable = of('Hello').pipe(
  delay(2000),
  timeout(1000)
);
observable.subscribe({
  next: value => console.log(value),
  error: err => console.error('Error:', err.message)
});
// Output: Error: Timeout has occurred
\end{lstlisting}

\subsection{timeoutWith}
\texttt{timeoutWith} pozwala na przełączenie się na inny \texttt{Observable}, jeśli nie nadejdzie żadna wartość w określonym czasie.

\begin{lstlisting}[language=JavaScript, caption=Przykład użycia \texttt{timeoutWith}]
import { of, timeoutWith, timer } from 'rxjs';

const observable = of('Hello').pipe(
  delay(2000),
  timeoutWith(1000, timer(0, 500))
);
observable.subscribe(value => console.log(value));
// Output (co 500 ms): 0, 1, 2, ...
\end{lstlisting}

\section{Tworzenie własnych operatorów}
Tworzenie własnych operatorów polega na utworzeniu funkcji, która przyjmuje \texttt{Observable} jako argument i zwraca nowy \texttt{Observable}.

\begin{lstlisting}[language=JavaScript, caption=Przykład tworzenia własnego operatora]
import { Observable } from 'rxjs';

function myCustomOperator() {
  return function(source) {
    return new Observable(observer => {
      return source.subscribe({
        next(value) {
          observer.next(value * 2);
        },
        error(err) {
          observer.error(err);
        },
        complete() {
          observer.complete();
        }
      });
    });
  };
}

const observable = of(1, 2, 3).pipe(myCustomOperator());
observable.subscribe(value => console.log(value));
// Output: 2, 4, 6
\end{lstlisting}

\section{Podsumowanie}
RxJS dostarcza potężny zestaw narzędzi do pracy z asynchronicznymi strumieniami danych. Zrozumienie operatorów pozwala na tworzenie zaawansowanych logik przepływu danych w aplikacjach, umożliwiając bardziej efektywne i czytelne zarządzanie stanem oraz operacjami asynchronicznymi.

\end{document}

